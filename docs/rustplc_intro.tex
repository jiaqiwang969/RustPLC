\documentclass[12pt,a4paper]{article}
\usepackage{xeCJK}
\usepackage{fontspec}
\usepackage{geometry}
\usepackage{graphicx}
\usepackage{listings}
\usepackage{xcolor}
\usepackage{amsmath,amssymb}
\usepackage{tikz}
\usetikzlibrary{arrows.meta,positioning,shapes.geometric,fit,calc}
\usepackage{hyperref}
\usepackage{booktabs}
\usepackage{enumitem}
\usepackage{fancyvrb}

\geometry{left=2.5cm,right=2.5cm,top=2.5cm,bottom=2.5cm}

\setCJKmainfont{STSong}
\setCJKsansfont{STHeiti}
\setCJKmonofont{STFangsong}
\setmainfont{Times New Roman}
\setmonofont{Menlo}

\definecolor{codebg}{RGB}{245,245,245}
\definecolor{codeframe}{RGB}{200,200,200}
\definecolor{keyword}{RGB}{0,0,180}
\definecolor{comment}{RGB}{0,128,0}
\definecolor{string}{RGB}{163,21,21}
\definecolor{section}{RGB}{180,0,0}

\lstdefinelanguage{PLC}{
  keywords={device,action,wait,timeout,goto,step,task,on_complete,parallel,race,branch,then,allow_indefinite_wait,reason,safety,timing,causality},
  keywordstyle=\color{keyword}\bfseries,
  morecomment=[l]{\#},
  commentstyle=\color{comment}\itshape,
  morestring=[b]",
  stringstyle=\color{string},
  literate={->}{{\textcolor{red}{->}}}{2}
           {[topology]}{{\textcolor{section}{\textbf{[topology]}}}}{10}
           {[constraints]}{{\textcolor{section}{\textbf{[constraints]}}}}{13}
           {[tasks]}{{\textcolor{section}{\textbf{[tasks]}}}}{7},
  sensitive=true,
}

\lstset{
  basicstyle=\small\ttfamily,
  backgroundcolor=\color{codebg},
  frame=single,
  rulecolor=\color{codeframe},
  breaklines=true,
  breakatwhitespace=true,
  tabsize=4,
  showstringspaces=false,
  numbers=left,
  numberstyle=\tiny\color{gray},
  numbersep=8pt,
  xleftmargin=2em,
  framexleftmargin=1.5em,
}

\title{\textbf{RustPLC:基于形式化验证的工业控制编译器} \\[0.5em]
\large 第一性原理分析与实例演练}
\author{RustPLC 项目学习笔记}
\date{2026年2月}

\begin{document}
\maketitle
\tableofcontents
\newpage

%% ============================================================
\section{引言:为什么需要 RustPLC}

传统 PLC(可编程逻辑控制器)编程使用梯形图、结构化文本(ST)或功能块图(FBD),
程序员手动编写控制逻辑,然后通过测试来验证正确性。这种``先写后测''的模式存在根本性缺陷:

\begin{itemize}[leftmargin=2em]
  \item \textbf{测试只能发现 bug,不能证明没有 bug}——测试覆盖的是有限路径,而工业系统的状态空间是指数级的。
  \item \textbf{安全约束散落在代码各处}——互锁逻辑与业务逻辑混杂,难以审计。
  \item \textbf{时序问题难以复现}——传感器延迟、气缸行程时间等物理参数在仿真中容易被忽略。
\end{itemize}

RustPLC 的核心主张是:

\begin{quote}
\textit{不要编写控制程序——声明物理事实和安全意图,让编译器在代码运行之前证明它是安全的。}
\end{quote}

%% ============================================================
\section{第一性原理}

RustPLC 的设计建立在以下不可再分的基本原理之上:

\subsection{原理一:物理系统是有限状态的}

工业控制中的每个设备都有有限的状态集合。气缸只有``伸出''和``缩回''两个状态,
电磁阀只有``开''和``关'',传感器只有``检测到''和``未检测到''。
整个系统的状态空间是所有设备状态的笛卡尔积:

\[
\mathcal{S} = S_{\text{dev}_1} \times S_{\text{dev}_2} \times \cdots \times S_{\text{dev}_n} \times S_{\text{ctrl}}
\]

其中 $S_{\text{ctrl}}$ 是控制状态机的状态集。由于每个因子都是有限的,
$\mathcal{S}$ 也是有限的——这意味着我们可以用穷举或归纳的方式证明性质。

\subsection{原理二:安全性是状态空间上的不变量}

``A缸和B缸不能同时伸出''这样的安全约束,本质上是对状态空间的一个划分:

\[
\mathcal{S}_{\text{safe}} = \{ s \in \mathcal{S} \mid \neg(\text{cyl\_A.extended} \wedge \text{cyl\_B.extended}) \}
\]

安全验证的任务就是证明:从初始状态出发,沿任意合法转移路径,系统永远停留在 $\mathcal{S}_{\text{safe}}$ 中。
这是一个经典的\textbf{模型检验}(Model Checking)问题。

\subsection{原理三:因果律决定信号传播路径}

物理信号不能凭空出现。一个传感器要检测到气缸伸出,必须存在完整的因果链:

\[
\text{数字输出} \xrightarrow{\text{电气}} \text{电磁阀} \xrightarrow{\text{气动}} \text{气缸} \xrightarrow{\text{物理}} \text{传感器}
\]

如果拓扑图中这条路径断开,程序逻辑再正确也无法工作。

\subsection{原理四:活性保证系统不会``卡死''}

一个安全但永远不动的系统没有价值。活性(Liveness)要求每个非终态都有出路:
要么有超时跳转,要么有明确的``允许无限等待''标记(如等待人工按钮)。

\subsection{原理五:时序约束来自物理定律}

电磁阀有响应时间,气缸有行程时间,电机有加速斜坡。这些不是软件参数,
而是物理常数。编译器必须沿拓扑连接链累加这些时间,计算最坏情况关键路径,
并与声明的时序约束比较。

%% ============================================================
\section{编译器架构}

RustPLC 采用经典的多阶段编译器设计,如图~\ref{fig:pipeline} 所示。

\begin{figure}[htbp]
\centering
\begin{tikzpicture}[
  stage/.style={rectangle, draw, rounded corners, minimum width=3.5cm, minimum height=1cm,
                fill=blue!8, font=\small\bfseries},
  data/.style={rectangle, draw, dashed, minimum width=3cm, minimum height=0.7cm,
               fill=yellow!10, font=\small},
  arr/.style={-{Stealth[length=3mm]}, thick},
  node distance=1.2cm
]
  \node[stage] (src) {.plc 源文件};
  \node[stage, below=of src] (parse) {解析器 (pest PEG)};
  \node[data, right=2cm of parse] (ast) {AST};
  \node[stage, below=of parse] (sem) {语义分析};
  \node[data, right=2cm of sem] (ir) {IR 中间表示};
  \node[stage, below=of sem] (ver) {形式化验证引擎};
  \node[data, right=2cm of ver] (report) {验证报告};
  \node[stage, below=of ver] (out) {JSON 输出};

  \draw[arr] (src) -- (parse);
  \draw[arr] (parse) -- (sem);
  \draw[arr] (parse) -- (ast);
  \draw[arr] (sem) -- (ver);
  \draw[arr] (sem) -- (ir);
  \draw[arr] (ver) -- (out);
  \draw[arr] (ver) -- (report);
\end{tikzpicture}
\caption{RustPLC 编译流水线}
\label{fig:pipeline}
\end{figure}

\subsection{阶段一:解析(Parser)}

使用 \texttt{pest} PEG 解析器,将 \texttt{.plc} 源文件转换为抽象语法树(AST)。
语法文件 \texttt{plc.pest} 定义了 150 条规则,覆盖三大段落:

\begin{itemize}
  \item \texttt{[topology]}——设备声明与物理连接
  \item \texttt{[constraints]}——安全/时序/因果约束
  \item \texttt{[tasks]}——控制任务与步骤(状态机)
\end{itemize}

\subsection{阶段二:语义分析(Semantic Analysis)}

将 AST 降低为四种中间表示:

\begin{table}[htbp]
\centering
\begin{tabular}{lll}
\toprule
\textbf{IR 结构} & \textbf{数据结构} & \textbf{用途} \\
\midrule
拓扑图 & \texttt{petgraph::DiGraph} & 设备连接关系 \\
状态机 & 状态 + 转移 + 守卫 & 控制流建模 \\
约束集 & 安全/时序/因果规则 & 验证目标 \\
时序模型 & 动作$\to$时间区间映射 & 时序分析 \\
\bottomrule
\end{tabular}
\caption{四种中间表示}
\end{table}

\subsection{阶段三:形式化验证}

四个独立的验证引擎并行运行:

\begin{enumerate}
  \item \textbf{安全检查器}——有界模型检验(BMC)+ $k$-归纳,可选 Z3 SMT 求解器
  \item \textbf{活性检查器}——强连通分量(SCC)分析 + 死锁检测
  \item \textbf{时序检查器}——最坏情况关键路径分析
  \item \textbf{因果检查器}——拓扑图 BFS 可达性验证
\end{enumerate}

%% ============================================================
\section{DSL 语言设计}

每个 \texttt{.plc} 文件由三个段落组成。下面以``传送带冲压系统''为例逐段讲解。

\subsection{拓扑段 \texttt{[topology]}}

声明物理设备及其连接关系:

\begin{lstlisting}[language=PLC]
[topology]

device Y0: digital_output
device Y1: digital_output
device X0: digital_input
device X1: digital_input
device X2: digital_input
device X3: digital_input

device start_button: digital_input {
    connected_to: X3
    debounce: 20ms
}

device conveyor_motor: motor {
    connected_to: Y0
    rated_speed: 30rpm
    ramp_time: 100ms
}

device stamp_valve: solenoid_valve {
    connected_to: Y1
    response_time: 15ms
}

device stamp_head: cylinder {
    connected_to: stamp_valve
    stroke_time: 250ms
    retract_time: 200ms
}

device sensor_in_position: sensor {
    type: proximity
    connected_to: X0
    detects: conveyor_motor.position_A
}

device sensor_stamp_down: sensor {
    type: magnetic
    connected_to: X1
    detects: stamp_head.extended
}

device sensor_stamp_up: sensor {
    type: magnetic
    connected_to: X2
    detects: stamp_head.retracted
}
\end{lstlisting}

\texttt{connected\_to} 表示上游连接:\texttt{stamp\_head} 连接到 \texttt{stamp\_valve},
意味着电磁阀驱动气缸。编译器据此构建有向拓扑图。

\subsection{约束段 \texttt{[constraints]}}

声明安全、时序和因果约束:

\begin{lstlisting}[language=PLC]
[constraints]

safety: stamp_head.extended conflicts_with conveyor_motor.on
    reason: "冲压头下压时传送带不能运行"

timing: task.cycle must_complete_within 3000ms
    reason: "单个冲压周期不应超过3秒"

causality: Y1 -> stamp_valve -> stamp_head -> sensor_stamp_down
causality: Y1 -> stamp_valve -> stamp_head -> sensor_stamp_up
\end{lstlisting}

\begin{itemize}
  \item \textbf{safety}:\texttt{conflicts\_with} 表示两个状态互斥,\texttt{requires} 表示蕴含。
  \item \textbf{timing}:\texttt{must\_complete\_within} 设定上界,\texttt{must\_start\_after} 设定下界。
  \item \textbf{causality}:用 \texttt{->} 链声明信号传播路径。
\end{itemize}

\subsection{任务段 \texttt{[tasks]}}

定义控制逻辑为状态机:

\begin{lstlisting}[language=PLC]
[tasks]

task cycle:
    step feed:
        action: set conveyor_motor on
        wait: sensor_in_position == true
        timeout: 1500ms -> goto fault_handler
    step stop_belt:
        action: set conveyor_motor off
    step press_down:
        action: extend stamp_head
        wait: sensor_stamp_down == true
        timeout: 500ms -> goto fault_handler
    step press_up:
        action: retract stamp_head
        wait: sensor_stamp_up == true
        timeout: 500ms -> goto fault_handler
    on_complete: goto ready

task fault_handler:
    step emergency:
        action: retract stamp_head
        action: set conveyor_motor off
    step report:
        action: log "冲压系统故障:动作超时"
    on_complete: goto ready

task ready:
    step wait_start:
        wait: start_button == true
        allow_indefinite_wait: true
    on_complete: goto cycle
\end{lstlisting}

%% ============================================================
\section{验证引擎详解}

\subsection{安全检查器:有界模型检验}

安全检查器的核心算法是\textbf{有界模型检验}(Bounded Model Checking, BMC),
辅以 $k$-\textbf{归纳}($k$-induction)来获得完备证明。

\subsubsection{状态空间建模}

系统的具体状态是一个元组:
\[
\sigma = (q, d_1, d_2, \ldots, d_n)
\]
其中 $q \in Q$ 是控制状态机的当前状态,$d_i \in D_i$ 是第 $i$ 个设备的当前状态。

\subsubsection{BMC 搜索}

从初始状态 $\sigma_0$ 出发,BFS 展开所有可达状态,直到深度 $k$:

\[
\text{Reach}_k = \{ \sigma \mid \exists \sigma_0 \xrightarrow{t_1} \sigma_1 \xrightarrow{t_2} \cdots \xrightarrow{t_k} \sigma, \; \sigma_0 \in \mathcal{I} \}
\]

对每个 \texttt{conflicts\_with} 约束 $\phi$,检查:
\[
\forall \sigma \in \text{Reach}_k : \sigma \models \neg\phi
\]

\subsubsection{$k$-归纳}

如果 BFS 在深度 $k$ 时状态空间已收敛(无新状态),则获得\textbf{完备证明}:
\[
\text{Reach}_k = \text{Reach}_{k+1} \implies \text{Reach}_k = \text{Reach}_\infty
\]

在我们的传送带冲压示例中,安全检查器在深度 8 获得完备证明,
证明了``冲压头下压时传送带不会运行''这一安全性质。

\subsection{活性检查器:死锁检测}

活性检查器执行四项检查:

\begin{enumerate}
  \item \textbf{无超时等待}:每个 \texttt{wait} 必须有 \texttt{timeout} 或 \texttt{allow\_indefinite\_wait}。
  \item \textbf{不可达声明验证}:标记为 \texttt{unreachable} 的 \texttt{on\_complete} 必须所有路径都跳走。
  \item \textbf{零出度检测}:非终态必须有出边。
  \item \textbf{SCC 分析}:使用 Kosaraju 算法检测可能困住执行的环路。
\end{enumerate}

\subsection{时序检查器:关键路径分析}

时序检查器沿拓扑连接链累加设备物理参数,计算最坏情况执行时间。

以冲压周期为例:

\begin{table}[htbp]
\centering
\begin{tabular}{llr}
\toprule
\textbf{步骤} & \textbf{关键动作} & \textbf{最坏时间} \\
\midrule
feed & 电机启动 + 等待到位 & 1500ms(超时上界) \\
stop\_belt & 电机停止 & 100ms \\
press\_down & 电磁阀响应 + 气缸行程 & 500ms(超时上界) \\
press\_up & 气缸回程 & 500ms(超时上界) \\
\midrule
\textbf{总计} & & \textbf{2600ms} \\
\bottomrule
\end{tabular}
\caption{冲压周期最坏情况时序分析}
\end{table}

$2600\text{ms} < 3000\text{ms}$,时序约束满足。

\subsection{因果检查器:信号可达性}

因果检查器在拓扑图上执行 BFS,验证声明的因果链中每一跳都有对应的物理连接:

\[
\text{Y1} \xrightarrow{\text{electrical}} \text{stamp\_valve} \xrightarrow{\text{pneumatic}} \text{stamp\_head} \xrightarrow{\text{logical}} \text{sensor\_stamp\_down}
\]

如果任何一跳断开(例如 \texttt{connected\_to} 配置错误),编译器会报告断裂位置并给出修复建议。

%% ============================================================
\section{完整示例:传送带冲压系统}

\subsection{场景描述}

一条自动化产线上,工件由传送带送入冲压工位。传感器检测到工件到位后,
传送带停止,冲压头下压完成冲压,然后回位,传送带继续运行。

关键安全约束:\textbf{冲压头下压时传送带绝对不能运行},否则工件移位会导致冲压偏移,
损坏模具甚至造成安全事故。

\subsection{系统拓扑}

\begin{figure}[htbp]
\centering
\begin{tikzpicture}[
  dev/.style={rectangle, draw, rounded corners, minimum width=2.2cm, minimum height=0.8cm, font=\small},
  io/.style={dev, fill=green!15},
  act/.style={dev, fill=orange!15},
  sen/.style={dev, fill=cyan!15},
  conn/.style={-{Stealth[length=2.5mm]}, thick},
  label/.style={font=\tiny, midway, above},
  node distance=1.5cm and 2cm
]
  \node[io] (y0) {Y0};
  \node[io, right=3cm of y0] (y1) {Y1};
  \node[act, below=of y0] (motor) {conveyor\_motor};
  \node[act, below=of y1] (valve) {stamp\_valve};
  \node[act, below=of valve] (head) {stamp\_head};
  \node[sen, below left=1.2cm and 0cm of motor] (spos) {sensor\_in\_position};
  \node[sen, below left=1.2cm and -0.5cm of head] (sdown) {sensor\_stamp\_down};
  \node[sen, below right=1.2cm and -0.5cm of head] (sup) {sensor\_stamp\_up};

  \draw[conn] (y0) -- node[label] {电气} (motor);
  \draw[conn] (y1) -- node[label] {电气} (valve);
  \draw[conn] (valve) -- node[label,right] {气动} (head);
  \draw[conn, dashed] (motor) -- node[label,left] {检测} (spos);
  \draw[conn, dashed] (head) -- node[label,left] {检测} (sdown);
  \draw[conn, dashed] (head) -- node[label,right] {检测} (sup);
\end{tikzpicture}
\caption{传送带冲压系统拓扑图}
\label{fig:topo}
\end{figure}

\subsection{状态机}

\begin{figure}[htbp]
\centering
\begin{tikzpicture}[
  state/.style={rectangle, draw, rounded corners, minimum width=2cm, minimum height=0.7cm,
                fill=blue!10, font=\small},
  arr/.style={-{Stealth[length=2.5mm]}, thick},
  timeout/.style={arr, dashed, red},
  node distance=1.5cm
]
  \node[state] (feed) {cycle.feed};
  \node[state, below=of feed] (stop) {cycle.stop\_belt};
  \node[state, below=of stop] (down) {cycle.press\_down};
  \node[state, below=of down] (up) {cycle.press\_up};
  \node[state, right=3cm of feed] (fault) {fault\_handler};
  \node[state, below=3cm of fault] (ready) {ready.wait\_start};

  \draw[arr] (feed) -- node[left,font=\tiny] {到位} (stop);
  \draw[arr] (stop) -- node[left,font=\tiny] {always} (down);
  \draw[arr] (down) -- node[left,font=\tiny] {下压到位} (up);
  \draw[arr] (up) -- node[left,font=\tiny] {回位到位} (ready);
  \draw[timeout] (feed) -- node[above,font=\tiny] {1500ms} (fault);
  \draw[timeout] (down.east) -- (fault);
  \draw[timeout] (up.east) -- (fault);
  \draw[arr] (fault) -- (ready);
  \draw[arr] (ready.west) -- +(-1.5,0) |- (feed.west);
\end{tikzpicture}
\caption{传送带冲压系统状态机(红色虚线为超时跳转)}
\label{fig:sm}
\end{figure}

\subsection{运行验证}

在项目根目录执行:

\begin{Verbatim}[frame=single,fontsize=\small]
$ cargo run -- examples/conveyor_stamp.plc
\end{Verbatim}

输出:

\begin{Verbatim}[frame=single,fontsize=\small]
验证通过:
  - Safety: 完备证明(深度 8)
  - Liveness: 通过
  - Timing: 通过
  - Causality: 通过
\end{Verbatim}

四项验证全部通过,含义如下:

\begin{itemize}
  \item \textbf{Safety 完备证明}:在深度 8 的状态空间搜索中,状态已收敛。
        数学上证明了在\textbf{任意}执行路径上,冲压头下压时传送带都不会运行。
  \item \textbf{Liveness 通过}:所有等待都有超时保护或明确的无限等待许可,
        不存在死锁。
  \item \textbf{Timing 通过}:最坏情况关键路径 2600ms $<$ 约束上界 3000ms。
  \item \textbf{Causality 通过}:从 Y1 到 sensor\_stamp\_down 和 sensor\_stamp\_up
        的信号传播路径在拓扑图中完整连通。
\end{itemize}

%% ============================================================
\section{技术栈与依赖}

\begin{table}[htbp]
\centering
\begin{tabular}{lll}
\toprule
\textbf{依赖} & \textbf{版本} & \textbf{用途} \\
\midrule
\texttt{pest} & 2.7 & PEG 解析器生成器 \\
\texttt{petgraph} & 0.6 & 图数据结构与算法(SCC, BFS) \\
\texttt{serde} & 1.0 & JSON 序列化 \\
\texttt{thiserror} & 1.0 & 错误类型派生 \\
\texttt{z3}(可选) & 0.12 & SMT 求解器增强安全证明 \\
\bottomrule
\end{tabular}
\caption{RustPLC 依赖}
\end{table}

Rust 2024 Edition,总代码量约 7,700 行,52 个测试用例。

%% ============================================================
\section{与传统 PLC 编程的对比}

\begin{table}[htbp]
\centering
\begin{tabular}{lcc}
\toprule
\textbf{特性} & \textbf{传统 PLC} & \textbf{RustPLC} \\
\midrule
安全验证 & 运行时测试 & 编译时形式化证明 \\
互锁逻辑 & 手动编码 & 声明式约束 \\
时序分析 & 示波器实测 & 编译器自动计算 \\
因果追踪 & 人工审查 & 拓扑图自动验证 \\
死锁检测 & 长时间运行观察 & SCC 静态分析 \\
错误定位 & 调试器 & 精确行号 + 修复建议 \\
\bottomrule
\end{tabular}
\caption{传统 PLC 编程 vs RustPLC}
\end{table}

%% ============================================================
\section{深度追踪:一条安全约束如何变成数学证明}

前面我们从宏观上介绍了编译器架构和验证引擎。但``自动化验证''到底是怎么实现的?
本节以传送带冲压系统中的一条安全约束为例,逐函数、逐数据结构地追踪它从 DSL 文本
变成数学证明的完整路径。

我们追踪的目标是这一行:

\begin{Verbatim}[frame=single,fontsize=\small]
safety: stamp_head.extended conflicts_with conveyor_motor.on
\end{Verbatim}

整个过程经历 5 个阶段,涉及 12 个关键函数和 8 个数据结构。

\subsection{阶段一:文本 $\to$ AST(解析)}

\subsubsection{PEG 语法匹配}

\texttt{pest} 解析器首先用 \texttt{plc.pest} 中的语法规则匹配这行文本。
相关规则链为:

\begin{Verbatim}[frame=single,fontsize=\small]
constraint_declaration = { safety_constraint | timing_constraint | ... }

safety_constraint = {
    "safety" ~ ":" ~ state_reference ~ safety_relation ~ state_reference
    ~ reason_clause?
}

state_reference = @{ identifier ~ "." ~ identifier }
safety_relation = { "conflicts_with" | "requires" }
\end{Verbatim}

匹配结果是一棵 pest 语法树,其中 \texttt{state\_reference} 匹配到
\texttt{stamp\_head.extended} 和 \texttt{conveyor\_motor.on},
\texttt{safety\_relation} 匹配到 \texttt{conflicts\_with}。

\subsubsection{AST 构建}

\texttt{parse\_constraints\_section()} 遍历语法树,遇到 \texttt{safety\_constraint}
规则时调用 \texttt{parse\_safety\_constraint()}(\texttt{src/parser/mod.rs:206}):

\begin{lstlisting}[language={[ISO]C++},morekeywords={fn,let,for,match,Some,Ok,Pair,Rule}]
fn parse_safety_constraint(pair: Pair<Rule>)
    -> Result<SafetyConstraint, PlcError>
{
    let line = line_of(&pair);
    let mut left = None;
    let mut relation = None;
    let mut right = None;
    let mut reason = None;

    for part in pair.into_inner() {
        match part.as_rule() {
            Rule::state_reference if left.is_none()
                => left = Some(parse_state_reference(part)?),
            Rule::safety_relation
                => relation = Some(parse_safety_relation(part)?),
            Rule::state_reference
                => right = Some(parse_state_reference(part)?),
            Rule::reason_clause
                => reason = Some(parse_reason_clause(part)?),
            _ => {}
        }
    }

    Ok(SafetyConstraint { line, left, relation, right, reason })
}
\end{lstlisting}

产出的 AST 节点为:

\begin{Verbatim}[frame=single,fontsize=\small]
SafetyConstraint {
    line: 47,
    left:  StateReference { device: "stamp_head",     state: "extended" },
    relation: ConflictsWith,
    right: StateReference { device: "conveyor_motor", state: "on" },
    reason: Some("冲压头下压时传送带不能运行...")
}
\end{Verbatim}

\subsection{阶段二:AST $\to$ IR(语义分析)}

\texttt{build\_constraint\_set\_from\_ast()}(\texttt{src/semantic/mod.rs:104})
接收整个 AST,对每条 safety 约束执行两步操作:

\subsubsection{步骤 2a:验证引用合法性}

调用 \texttt{validate\_state\_reference()} 检查:
\begin{itemize}
  \item \texttt{stamp\_head} 是否在 \texttt{[topology]} 中声明过?——是,类型为 \texttt{cylinder}。
  \item \texttt{extended} 是否是 \texttt{cylinder} 的合法状态?——是,气缸的默认状态域为 \texttt{\{extended, retracted\}}。
  \item \texttt{conveyor\_motor} 是否存在?——是,类型为 \texttt{motor}。
  \item \texttt{on} 是否是 \texttt{motor} 的合法状态?——是,电机的默认状态域为 \texttt{\{on, off\}}。
\end{itemize}

如果任何一项检查失败,编译器会报告精确行号和修复建议,编译终止。

\subsubsection{步骤 2b:降低为 IR}

验证通过后,AST 节点被转换为 IR 结构:

\begin{Verbatim}[frame=single,fontsize=\small]
SafetyRule {
    left:     StateExpr { device: "stamp_head",     state: "extended" },
    relation: ConflictsWith,
    right:    StateExpr { device: "conveyor_motor", state: "on" },
    reason:   Some("冲压头下压时传送带不能运行...")
}
\end{Verbatim}

这个 \texttt{SafetyRule} 被存入 \texttt{ConstraintSet.safety} 向量,
等待验证引擎消费。

\subsection{阶段三:IR $\to$ SafetyModel(模型构建)}

安全检查器的入口是 \texttt{verify\_safety()}(\texttt{src/verification/safety.rs:124}),
它首先调用 \texttt{SafetyModel::from\_inputs()} 构建验证模型。这是最关键的一步。

\subsubsection{步骤 3a:设备域枚举}

\texttt{collect\_device\_domains()}(\texttt{safety.rs:358})为每个设备建立有限状态域:

\begin{Verbatim}[frame=single,fontsize=\small]
DeviceDomain { name: "stamp_head",     states: ["extended","retracted"],
               default_state: 1 }  // 初始为 retracted
DeviceDomain { name: "conveyor_motor", states: ["on","off"],
               default_state: 1 }  // 初始为 off
DeviceDomain { name: "stamp_valve",    states: ["on","off"],
               default_state: 1 }
...(共 13 个设备)
\end{Verbatim}

同时建立两级索引:
\begin{itemize}
  \item \texttt{device\_index}: \texttt{"stamp\_head" $\to$ 9, "conveyor\_motor" $\to$ 7, ...}
  \item \texttt{device\_state\_index[9]}: \texttt{"extended" $\to$ 0, "retracted" $\to$ 1}
\end{itemize}

\subsubsection{步骤 3b:转移效果提取}

\texttt{transition\_effects()}(\texttt{safety.rs:430})将状态机中每条转移的
\texttt{action} 映射为设备状态变更。例如:

\begin{Verbatim}[frame=single,fontsize=\small]
转移: cycle.feed -> cycle.stop_belt
  动作: set conveyor_motor on
  效果: { device_id=7 -> state_id=0 }  // conveyor_motor := on

转移: cycle.stop_belt -> cycle.press_down
  动作: set conveyor_motor off
  效果: { device_id=7 -> state_id=1 }  // conveyor_motor := off

转移: cycle.press_down -> cycle.press_up
  动作: extend stamp_head
  效果: { device_id=9 -> state_id=0 }  // stamp_head := extended
\end{Verbatim}

关键观察:\texttt{extend stamp\_head} 发生在 \texttt{press\_down $\to$ press\_up},
而此时 \texttt{conveyor\_motor} 已经在上一步被设为 \texttt{off}。
这正是安全约束能通过的原因——但编译器不靠``看'',它靠穷举证明。

\subsubsection{步骤 3c:搜索深度规划}

\texttt{scc\_minimum\_depth()}(\texttt{safety.rs:510})用 Kosaraju 算法
找到状态机中最大的强连通分量(SCC),取 $|\text{SCC}|+1$ 作为搜索深度下界。
最终深度取 $\max(\text{状态数}, \text{SCC深度})$。

在我们的例子中:状态机有 7 个状态(cycle 的 4 步 + fault\_handler 的 2 步 + ready),
存在 \texttt{ready $\to$ cycle $\to$ ready} 的环路,SCC 大小为 7,
所以搜索深度为 $\max(7, 7+1) = 8$。

\subsection{阶段四:BFS 穷举搜索}

\subsubsection{步骤 4a:约束绑定}

\texttt{bind\_rule()}(\texttt{safety.rs:584})将 IR 层的字符串约束绑定为数值索引:

\begin{Verbatim}[frame=single,fontsize=\small]
RuleBinding {
    left_device:  9,   // stamp_head
    left_state:   0,   // extended
    right_device: 7,   // conveyor_motor
    right_state:  0,   // on
}
\end{Verbatim}

这样后续的冲突检测只需比较两个整数,无需字符串操作。

\subsubsection{步骤 4b:BFS 展开}

\texttt{analyze\_rule()}(\texttt{safety.rs:609})是核心搜索函数。
它维护一个 BFS 队列和一个已访问状态集:

\begin{lstlisting}[language={[ISO]C++},morekeywords={fn,let,while,for,if}]
fn analyze_rule(model, rule, max_depth) -> SearchOutcome {
    let initial = initial_concrete_state(model);
    // initial = (control=0, devices=[1,1,1,...,1,1,...])
    //           所有设备处于默认状态(off/retracted)

    let mut queue = VecDeque::from([initial]);
    let mut visited = HashMap::new();

    while let Some(node) = queue.pop_front() {
        // 冲突检测:O(1) 整数比较
        if conflicts(&node.state, rule) {
            return counterexample(path);
        }

        if node.depth == max_depth { continue; }

        for edge in outgoing_edges(node) {
            let next = apply_edge(edge, &node.state);
            if !visited.contains(&next) {
                queue.push_back(next);
            }
        }
    }

    SearchOutcome { counterexample: None,
                    fully_explored: true }
}
\end{lstlisting}

\subsubsection{步骤 4c:冲突检测}

\texttt{conflicts()}(\texttt{safety.rs:703})极其简洁:

\begin{lstlisting}[language={[ISO]C++},morekeywords={fn}]
fn conflicts(state: &ConcreteState, rule: RuleBinding) -> bool {
    state.device_states[rule.left_device] == rule.left_state
    && state.device_states[rule.right_device] == rule.right_state
}
\end{lstlisting}

即:当前状态中 \texttt{stamp\_head} 是否为 \texttt{extended}(索引 0),
\textbf{同时} \texttt{conveyor\_motor} 是否为 \texttt{on}(索引 0)?

\subsubsection{搜索过程可视化}

以下是 BFS 展开的前几步(简化表示,仅显示两个关键设备的状态):

\begin{table}[htbp]
\centering
\small
\begin{tabular}{clccc}
\toprule
\textbf{深度} & \textbf{控制状态} & \textbf{stamp\_head} & \textbf{conveyor\_motor} & \textbf{冲突?} \\
\midrule
0 & cycle.feed & retracted & off & \textsf{No} \\
1 & cycle.stop\_belt & retracted & \textbf{on} & \textsf{No} \\
1 & fault\_handler.emergency & retracted & off & \textsf{No} \\
2 & cycle.press\_down & retracted & \textbf{off} & \textsf{No} \\
3 & cycle.press\_up & \textbf{extended} & \textbf{off} & \textsf{No} \\
3 & fault\_handler.emergency & retracted & off & \textsf{No} \\
4 & ready.wait\_start & retracted & off & \textsf{No} \\
$\vdots$ & $\vdots$ & $\vdots$ & $\vdots$ & $\vdots$ \\
\bottomrule
\end{tabular}
\caption{BFS 搜索过程(关键设备状态追踪)}
\end{table}

注意深度 3 的关键行:\texttt{stamp\_head} 变为 \texttt{extended} 时,
\texttt{conveyor\_motor} 已经是 \texttt{off}——因为 \texttt{stop\_belt} 步骤
在 \texttt{press\_down} 之前执行了 \texttt{set conveyor\_motor off}。

BFS 在深度 8 穷尽了所有可达状态,没有发现任何一个状态同时满足
\texttt{stamp\_head=extended} 且 \texttt{conveyor\_motor=on}。

\subsection{阶段五:证明判定与报告}

\subsubsection{收敛判定}

回到 \texttt{analyze\_rule()} 的返回值:

\begin{Verbatim}[frame=single,fontsize=\small]
SearchOutcome {
    counterexample: None,     // 未找到反例
    fully_explored: true,     // 深度 8 内无新状态
}
\end{Verbatim}

\texttt{fully\_explored = true} 意味着 $\text{Reach}_8 = \text{Reach}_9$,
即状态空间已收敛。这等价于 $k$-归纳的归纳步成立:

\[
\text{Reach}_8 = \text{Reach}_\infty \implies \forall \sigma \in \text{Reach}_\infty : \neg(\text{stamp\_head.extended} \wedge \text{conveyor\_motor.on})
\]

\subsubsection{报告生成}

\texttt{verify\_safety\_with\_config()}(\texttt{safety.rs:132})汇总所有约束的检查结果:

\begin{Verbatim}[frame=single,fontsize=\small]
SafetyReport {
    level: Complete,          // 完备证明
    explored_depth: 8,
    warnings: [],             // 无警告
}
\end{Verbatim}

最终由 \texttt{verify\_all()}(\texttt{verification/mod.rs:54})将四个引擎的结果
汇总为 \texttt{VerificationSummary},输出到 stderr:

\begin{Verbatim}[frame=single,fontsize=\small]
验证通过:
  - Safety: 完备证明(深度 8)
\end{Verbatim}

\subsection{完整数据流总结}

\begin{figure}[htbp]
\centering
\begin{tikzpicture}[
  box/.style={rectangle, draw, rounded corners, minimum width=4cm, minimum height=0.8cm,
              font=\small, fill=blue!5},
  func/.style={rectangle, draw, minimum width=3.5cm, minimum height=0.6cm,
               font=\scriptsize\ttfamily, fill=green!8},
  data/.style={rectangle, draw, dashed, minimum width=3.5cm, minimum height=0.6cm,
               font=\scriptsize, fill=yellow!8},
  arr/.style={-{Stealth[length=2.5mm]}, thick},
  node distance=0.8cm
]
  \node[data] (src) {DSL 文本: \texttt{safety: stamp\_head.extended ...}};
  \node[func, below=of src] (f1) {parse\_safety\_constraint()};
  \node[data, below=of f1] (ast) {AST::SafetyConstraint};
  \node[func, below=of ast] (f2) {build\_constraint\_set\_from\_ast()};
  \node[data, below=of f2] (ir) {IR::SafetyRule};
  \node[func, below=of ir] (f3) {SafetyModel::from\_inputs()};
  \node[data, below=of f3] (model) {SafetyModel + RuleBinding};
  \node[func, below=of model] (f4) {analyze\_rule() — BFS};
  \node[data, below=of f4] (outcome) {SearchOutcome: 无反例 + 已收敛};
  \node[box, below=of outcome] (proof) {完备证明 (Complete)};

  \draw[arr] (src) -- (f1);
  \draw[arr] (f1) -- (ast);
  \draw[arr] (ast) -- (f2);
  \draw[arr] (f2) -- (ir);
  \draw[arr] (ir) -- (f3);
  \draw[arr] (f3) -- (model);
  \draw[arr] (model) -- (f4);
  \draw[arr] (f4) -- (outcome);
  \draw[arr] (outcome) -- (proof);
\end{tikzpicture}
\caption{安全约束从文本到证明的完整数据流}
\label{fig:dataflow}
\end{figure}

\subsection{反例:如果约束被违反会怎样?}

为了对比,我们看看如果把 \texttt{stop\_belt} 步骤删掉(传送带不停就冲压),
会发生什么。编译器会在 BFS 中发现一条违反路径:

\begin{Verbatim}[frame=single,fontsize=\small]
ERROR [safety] 状态互斥违反
  位置: <input>:47:1
  约束: stamp_head.extended conflicts_with conveyor_motor.on
  违反路径:
    1. 初始状态 cycle.feed
    2. cycle.feed --[condition;动作: set conveyor_motor on]--> cycle.press_down
    3. cycle.press_down --[condition;动作: extend stamp_head]--> cycle.press_up
    4. 在 cycle.press_up 检测到冲突:stamp_head.extended 与 conveyor_motor.on 同时为真
  建议: 请在触发 conveyor_motor.on 之前确保 stamp_head.extended 已复位,
        或调整并行/跳转逻辑避免两者同时成立
\end{Verbatim}

\texttt{render\_path()}(\texttt{safety.rs:708})沿 BFS 的 \texttt{parent} 指针
回溯,重建从初始状态到违反状态的完整路径,让工程师能精确定位问题。

%% ============================================================
\section{验证实验:6 个例子逐一验证四大引擎}

光讲原理不够,下面我们用 6 个可运行的 \texttt{.plc} 例子,逐一触发四个验证引擎的
通过和失败,亲眼看编译器怎么抓 bug。

所有例子均在 \texttt{examples/} 目录下,可直接运行:
\begin{Verbatim}[frame=single,fontsize=\small]
$ cargo run -- examples/ex1_safety_pass.plc
\end{Verbatim}

%% ----------------------------------------
\subsection{例 1:安全性验证——顺序执行,通过}

场景:双气缸交替冲压,声明 \texttt{cyl\_A.extended conflicts\_with cyl\_B.extended}。
控制逻辑是\textbf{顺序执行}:先伸 A、缩 A,再伸 B、缩 B,不会同时伸出。

\begin{lstlisting}[language=PLC]
[constraints]
safety: cyl_A.extended conflicts_with cyl_B.extended
    reason: "两个气缸同时伸出会发生机械碰撞"

[tasks]
task main:
    step extend_A:
        action: extend cyl_A
        wait: sensor_A == true
        timeout: 500ms -> goto fault
    step retract_A:
        action: retract cyl_A
    step extend_B:
        action: extend cyl_B
        wait: sensor_B == true
        timeout: 500ms -> goto fault
    step retract_B:
        action: retract cyl_B
    on_complete: goto main
\end{lstlisting}

运行结果:

\begin{Verbatim}[frame=single,fontsize=\small]
验证通过:
  - Safety: 完备证明(深度 7)
  - Liveness: 通过
  - Timing: 通过
  - Causality: 通过
\end{Verbatim}

编译器在深度 7 穷尽了所有可达状态,数学上证明了两个气缸不可能同时伸出。

%% ----------------------------------------
\subsection{例 2:安全性验证——并行执行,失败}

同样的设备和约束,但把控制逻辑改成 \texttt{parallel} 同时伸出两个气缸:

\begin{lstlisting}[language=PLC]
[tasks]
task main:
    step both:
        parallel:
            branch_A:
                action: extend cyl_A
            branch_B:
                action: extend cyl_B
\end{lstlisting}

运行结果:

\begin{Verbatim}[frame=single,fontsize=\small]
ERROR [safety] 验证失败
  位置: <input>:23:1
  原因: 约束 cyl_A.extended conflicts_with cyl_B.extended
        在可达路径上可同时成立
  分析: 违反路径:
        初始状态 main.both
        -> main.both__parallel_1_fork
        -> main.both__parallel_1_branch_1
        -> main.both__parallel_1_join
        在 join 检测到冲突:cyl_A.extended 与 cyl_B.extended 同时为真
  建议: 请在触发 cyl_B.extended 之前确保 cyl_A.extended 已复位
\end{Verbatim}

编译器通过 BFS 找到了一条从初始状态到冲突状态的具体路径:
\texttt{parallel} 的两个分支在 \texttt{join} 点汇合时,两个气缸都处于伸出状态。

\textbf{对比}:例 1 和例 2 的设备、约束完全相同,唯一区别是控制逻辑。
顺序执行通过,并行执行被拦截——这就是形式化验证的价值。

%% ----------------------------------------
\subsection{例 3:活性验证——缺少超时,失败}

场景:气缸伸出后等待传感器信号,但\textbf{没有 timeout},也没有 \texttt{allow\_indefinite\_wait}。

\begin{lstlisting}[language=PLC]
[tasks]
task main:
    step go:
        action: extend cyl
        wait: sensor == true
        # 没有 timeout!传感器坏了就永远卡死
    step back:
        action: retract cyl
    on_complete: goto main
\end{lstlisting}

运行结果:

\begin{Verbatim}[frame=single,fontsize=\small]
ERROR [liveness] 验证失败
  位置: <input>:21:1
  原因: task main.go 的 wait 条件 `sensor == true` 缺少 timeout 分支,
        且未设置 allow_indefinite_wait
  分析: 若传感器信号长期不满足(线路故障/执行器卡滞/设备离线),
        控制逻辑会永久停留在该等待点
  建议: 请为该 step 添加 `timeout: <时长> -> goto <恢复 task>`

ERROR [liveness] 验证失败
  位置: <input>:21:1
  原因: 检测到强连通分量 [main.back, main.go] 不包含 timeout
        或 allow_indefinite_wait 出边
  分析: 一旦进入该循环,若条件长期不满足,流程会在环内反复执行
        且没有超时/人工等待豁免出口
\end{Verbatim}

活性检查器报了两个问题:
\begin{enumerate}
  \item 单点问题:\texttt{wait} 没有 \texttt{timeout},传感器坏了就卡死。
  \item 结构问题:\texttt{main.go $\to$ main.back $\to$ main.go} 形成 SCC 环路,
        环内没有任何超时出口。
\end{enumerate}

修复方法:加一行 \texttt{timeout: 500ms -> goto fault}。

%% ----------------------------------------
\subsection{例 4:时序验证——物理上不可能,失败}

场景:要求 100ms 内完成,但光气缸行程就要 300ms。

\begin{lstlisting}[language=PLC]
[topology]
device valve: solenoid_valve { connected_to: Y0, response_time: 20ms }
device cyl: cylinder { connected_to: valve, stroke_time: 300ms,
                       retract_time: 300ms }

[constraints]
timing: task.main must_complete_within 100ms
    reason: "要求 100ms 完成"

[tasks]
task main:
    step go:
        action: extend cyl
        wait: sensor == true
        timeout: 500ms -> goto err
    on_complete: goto main
\end{lstlisting}

运行结果:

\begin{Verbatim}[frame=single,fontsize=\small]
ERROR [timing] 验证失败
  位置: <input>:17:1
  原因: task.main must_complete_within 100ms;
        最坏情况下无法在 100ms 内完成,当前关键路径为 500ms
  分析: task main 的最坏关键路径时间 = 500ms(顺序 step 累加)
  建议: 请放宽 must_complete_within 阈值,或缩短动作响应/行程时间
\end{Verbatim}

时序检查器沿拓扑链累加物理参数:电磁阀响应 20ms + 气缸行程 300ms = 320ms,
加上 timeout 上界 500ms,远超 100ms 约束。这不是软件 bug,是物理定律不允许。

%% ----------------------------------------
\subsection{例 5:因果验证——接线错误,失败}

场景:\texttt{valve} 实际连接的是 \texttt{cyl\_A},但因果链声明写成了 \texttt{cyl\_B}。

\begin{lstlisting}[language=PLC]
[topology]
device valve: solenoid_valve { connected_to: Y0 }
device cyl_A: cylinder { connected_to: valve }
device cyl_B: cylinder { connected_to: valve }
device sensor: sensor { connected_to: X0, detects: cyl_A.extended }

[constraints]
# 错误:valve 连的是 cyl_A,但这里写了 cyl_B
causality: Y0 -> valve -> cyl_B -> sensor
\end{lstlisting}

运行结果:

\begin{Verbatim}[frame=single,fontsize=\small]
ERROR [causality] 验证失败
  位置: <input>:18:1
  原因: 检测到因果链断裂:cyl_B -> sensor
  分析: 期望链路: Y0 -> valve -> cyl_B -> sensor
        实际链路: Y0 -> valve -> cyl_B -> ???
  建议: 请在 [topology] 中检查 sensor 的 connected_to / detects 配置,
        确保链路 cyl_B -> sensor 可达
\end{Verbatim}

因果检查器在拓扑图上做 BFS,发现 \texttt{cyl\_B} 到 \texttt{sensor} 之间没有连接。
\texttt{sensor} 的 \texttt{detects} 写的是 \texttt{cyl\_A.extended},不是 \texttt{cyl\_B}。
这种接线错误在传统 PLC 编程中只能靠现场调试发现,RustPLC 在编译时就拦截了。

%% ----------------------------------------
\subsection{例 6:四项全部通过——完整工业场景}

场景:单气缸往复,安全 + 时序 + 因果 + 活性全覆盖。

\begin{lstlisting}[language=PLC]
[constraints]
safety: cyl.extended conflicts_with cyl.retracted
    reason: "气缸不可能同时处于伸出和缩回状态"
timing: task.work must_complete_within 2000ms
    reason: "单次往复不超过 2 秒"
causality: Y0 -> valve -> cyl -> sensor_ext
causality: Y0 -> valve -> cyl -> sensor_ret

[tasks]
task work:
    step push:
        action: extend cyl
        wait: sensor_ext == true
        timeout: 600ms -> goto fault
    step pull:
        action: retract cyl
        wait: sensor_ret == true
        timeout: 600ms -> goto fault
    on_complete: goto idle

task fault:
    step safe:
        action: retract cyl
    step alarm:
        action: log "气缸动作超时"
    on_complete: goto idle

task idle:
    step wait:
        wait: start_btn == true
        allow_indefinite_wait: true
    on_complete: goto work
\end{lstlisting}

运行结果:

\begin{Verbatim}[frame=single,fontsize=\small]
验证通过:
  - Safety: 完备证明(深度 6)
  - Liveness: 通过
  - Timing: 通过
  - Causality: 通过
\end{Verbatim}

%% ----------------------------------------
\subsection{实验总结}

\begin{table}[htbp]
\centering
\begin{tabular}{clcccc}
\toprule
\textbf{例} & \textbf{场景} & \textbf{Safety} & \textbf{Liveness} & \textbf{Timing} & \textbf{Causality} \\
\midrule
1 & 双缸顺序执行 & \checkmark & \checkmark & \checkmark & \checkmark \\
2 & 双缸并行执行 & \textsf{FAIL} & \textsf{FAIL} & -- & -- \\
3 & 等待无超时 & -- & \textsf{FAIL} & -- & -- \\
4 & 时序过紧 & -- & -- & \textsf{FAIL} & -- \\
5 & 因果链断裂 & -- & -- & -- & \textsf{FAIL} \\
6 & 单缸完整场景 & \checkmark & \checkmark & \checkmark & \checkmark \\
\bottomrule
\end{tabular}
\caption{6 个例子的验证结果汇总}
\end{table}

每个验证引擎都被至少一个正例和一个反例覆盖。
正例证明编译器不会误报,反例证明编译器能精确定位问题并给出修复建议。

%% ============================================================
\section{总结}

RustPLC 将工业控制编程从``编写程序然后测试''提升到``声明意图然后证明''。
它的五个第一性原理——有限状态、安全不变量、因果律、活性、物理时序——
构成了一个完整的形式化验证框架。

通过 6 个可运行的例子,我们验证了:
\begin{itemize}
  \item 安全检查器能区分顺序执行(安全)和并行执行(冲突),给出完备证明或违反路径;
  \item 活性检查器能发现缺少超时的等待点和 SCC 死循环;
  \item 时序检查器能根据物理参数计算最坏关键路径,拦截不可能满足的时序约束;
  \item 因果检查器能在拓扑图上发现接线错误,精确报告断裂位置。
\end{itemize}

这正是 RustPLC 的价值所在:\textbf{让安全成为编译器的责任,而非程序员的负担}。

\end{document}
