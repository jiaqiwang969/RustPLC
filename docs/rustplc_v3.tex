\documentclass[12pt,a4paper]{article}
\usepackage{xeCJK}
\usepackage{fontspec}
\usepackage{geometry}
\usepackage{graphicx}
\usepackage{listings}
\usepackage{xcolor}
\usepackage{amsmath,amssymb}
\usepackage{tikz}
\usetikzlibrary{arrows.meta,positioning,shapes.geometric,fit,calc}
\usepackage{hyperref}
\usepackage{booktabs}
\usepackage{enumitem}
\usepackage{fancyvrb}

\geometry{left=2.5cm,right=2.5cm,top=2.5cm,bottom=2.5cm}

\setCJKmainfont{STSong}
\setCJKsansfont{STHeiti}
\setCJKmonofont{STFangsong}
\setmainfont{Times New Roman}
\setmonofont{Menlo}

\definecolor{codebg}{RGB}{245,245,245}
\definecolor{codeframe}{RGB}{200,200,200}
\definecolor{keyword}{RGB}{0,0,180}
\definecolor{comment}{RGB}{0,128,0}
\definecolor{string}{RGB}{163,21,21}
\definecolor{section}{RGB}{180,0,0}

\lstdefinelanguage{PLC}{
  keywords={device,action,wait,timeout,goto,step,task,on_complete,parallel,race,branch,then,allow_indefinite_wait,reason,safety,timing,causality},
  keywordstyle=\color{keyword}\bfseries,
  morecomment=[l]{\#},
  commentstyle=\color{comment}\itshape,
  morestring=[b]",
  stringstyle=\color{string},
  literate={->}{{\textcolor{red}{->}}}{2}
           {[topology]}{{\textcolor{section}{\textbf{[topology]}}}}{10}
           {[constraints]}{{\textcolor{section}{\textbf{[constraints]}}}}{13}
           {[tasks]}{{\textcolor{section}{\textbf{[tasks]}}}}{7},
  sensitive=true,
}

\lstdefinelanguage{RustPLC}{
  keywords={fn,let,mut,pub,struct,enum,impl,match,if,else,for,while,loop,return,use,mod,crate,self,Self,true,false,Some,None,Ok,Err,Result,Option},
  keywordstyle=\color{keyword}\bfseries,
  morecomment=[l]{//},
  morecomment=[s]{/*}{*/},
  commentstyle=\color{comment}\itshape,
  morestring=[b]",
  stringstyle=\color{string},
  sensitive=true,
}

\lstset{
  basicstyle=\small\ttfamily,
  backgroundcolor=\color{codebg},
  frame=single,
  rulecolor=\color{codeframe},
  breaklines=true,
  breakatwhitespace=true,
  tabsize=4,
  showstringspaces=false,
  numbers=left,
  numberstyle=\tiny\color{gray},
  numbersep=8pt,
  xleftmargin=2em,
  framexleftmargin=1.5em,
}

\title{\textbf{RustPLC:工业控制数字孪生平台} \\[0.5em]
\large 从形式化验证到虚实混合闭环}
\author{RustPLC 项目技术文档 v3}
\date{2026年2月}

\begin{document}
\maketitle
\tableofcontents
\newpage

%% ============================================================
%% 第一部分:验证域 (BC1) —— 沿用 v2 内容
%% ============================================================

\part{验证域 (BC1):从 DSL 到形式化证明}

\section{引言:为什么需要 RustPLC}

传统 PLC(可编程逻辑控制器)编程使用梯形图、结构化文本(ST)或功能块图(FBD),
程序员手动编写控制逻辑,然后通过测试来验证正确性。这种``先写后测''的模式存在根本性缺陷:

\begin{itemize}[leftmargin=2em]
  \item \textbf{测试只能发现 bug,不能证明没有 bug}——测试覆盖的是有限路径,而工业系统的状态空间是指数级的。
  \item \textbf{安全约束散落在代码各处}——互锁逻辑与业务逻辑混杂,难以审计。
  \item \textbf{时序问题难以复现}——传感器延迟、气缸行程时间等物理参数在仿真中容易被忽略。
\end{itemize}

RustPLC 的核心主张是:

\begin{quote}
\textit{不要编写控制程序——声明物理事实和安全意图,让编译器在代码运行之前证明它是安全的。}
\end{quote}

但 RustPLC 不止于验证。它是一个完整的数字孪生平台,覆盖从源码到物理世界的完整闭环:

\begin{figure}[htbp]
\centering
\begin{tikzpicture}[
  box/.style={rectangle, draw, rounded corners, minimum width=3.5cm, minimum height=0.8cm,
              font=\small\bfseries, fill=blue!8},
  mode/.style={rectangle, draw, rounded corners, minimum width=3cm, minimum height=0.7cm,
               font=\small, fill=green!8},
  ext/.style={rectangle, draw, dashed, minimum width=3cm, minimum height=0.7cm,
              font=\small, fill=yellow!8},
  arr/.style={-{Stealth[length=3mm]}, thick},
  node distance=1cm
]
  \node[box] (plc) {.plc 源码};
  \node[box, below=of plc] (verify) {形式化验证};
  \node[box, below=of verify] (codegen) {代码生成};
  \node[box, below=of codegen] (binary) {Rust binary};

  \node[mode, below left=1.5cm and -0.5cm of binary] (modeA) {模式 A: SimBackend};
  \node[mode, below=1.5cm of binary] (modeB) {模式 B: 虚拟工厂};
  \node[mode, below right=1.5cm and -0.5cm of binary] (modeC) {模式 C: FPGA};

  \node[ext, below=0.8cm of modeB] (sim) {Gnucap + jtufem-rs};
  \node[ext, below=0.8cm of modeC] (fpga) {iCESugar-pro};

  \draw[arr] (plc) -- (verify);
  \draw[arr] (verify) -- (codegen);
  \draw[arr] (codegen) -- (binary);
  \draw[arr] (binary) -- (modeA);
  \draw[arr] (binary) -- (modeB);
  \draw[arr] (binary) -- (modeC);
  \draw[arr] (modeB) -- (sim);
  \draw[arr] (modeC) -- (fpga);
  \draw[arr, dashed] (sim.west) -- +(-1,0) |- node[left,font=\tiny,pos=0.25] {DI 回写} (modeB.west);
  \draw[arr, dashed] (fpga.east) -- +(1,0) |- node[right,font=\tiny,pos=0.25] {GPIO 回读} (modeC.east);
\end{tikzpicture}
\caption{RustPLC 完整闭环:从源码到物理世界}
\label{fig:closedloop}
\end{figure}

\section{第一性原理}

RustPLC 的设计建立在以下不可再分的基本原理之上:

\subsection{原理一:物理系统是有限状态的}

工业控制中的每个设备都有有限的状态集合。气缸只有``伸出''和``缩回''两个状态,
电磁阀只有``开''和``关'',传感器只有``检测到''和``未检测到''。
整个系统的状态空间是所有设备状态的笛卡尔积:

\[
\mathcal{S} = S_{\text{dev}_1} \times S_{\text{dev}_2} \times \cdots \times S_{\text{dev}_n} \times S_{\text{ctrl}}
\]

其中 $S_{\text{ctrl}}$ 是控制状态机的状态集。由于每个因子都是有限的,
$\mathcal{S}$ 也是有限的——这意味着我们可以用穷举或归纳的方式证明性质。

\subsection{原理二:安全性是状态空间上的不变量}

``A缸和B缸不能同时伸出''这样的安全约束,本质上是对状态空间的一个划分:

\[
\mathcal{S}_{\text{safe}} = \{ s \in \mathcal{S} \mid \neg(\text{cyl\_A.extended} \wedge \text{cyl\_B.extended}) \}
\]

安全验证的任务就是证明:从初始状态出发,沿任意合法转移路径,系统永远停留在 $\mathcal{S}_{\text{safe}}$ 中。
这是一个经典的\textbf{模型检验}(Model Checking)问题。

\subsection{原理三:因果律决定信号传播路径}

物理信号不能凭空出现。一个传感器要检测到气缸伸出,必须存在完整的因果链:

\[
\text{数字输出} \xrightarrow{\text{电气}} \text{电磁阀} \xrightarrow{\text{气动}} \text{气缸} \xrightarrow{\text{物理}} \text{传感器}
\]

如果拓扑图中这条路径断开,程序逻辑再正确也无法工作。

\subsection{原理四:活性保证系统不会``卡死''}

一个安全但永远不动的系统没有价值。活性(Liveness)要求每个非终态都有出路:
要么有超时跳转,要么有明确的``允许无限等待''标记(如等待人工按钮)。

\subsection{原理五:时序约束来自物理定律}

电磁阀有响应时间,气缸有行程时间,电机有加速斜坡。这些不是软件参数,
而是物理常数。编译器必须沿拓扑连接链累加这些时间,计算最坏情况关键路径,
并与声明的时序约束比较。

\section{编译器架构}

RustPLC 采用经典的多阶段编译器设计,如图~\ref{fig:pipeline} 所示。

\begin{figure}[htbp]
\centering
\begin{tikzpicture}[
  stage/.style={rectangle, draw, rounded corners, minimum width=3.5cm, minimum height=1cm,
                fill=blue!8, font=\small\bfseries},
  data/.style={rectangle, draw, dashed, minimum width=3cm, minimum height=0.7cm,
               fill=yellow!10, font=\small},
  arr/.style={-{Stealth[length=3mm]}, thick},
  node distance=1.2cm
]
  \node[stage] (src) {.plc 源文件};
  \node[stage, below=of src] (parse) {解析器 (pest PEG)};
  \node[data, right=2cm of parse] (ast) {AST};
  \node[stage, below=of parse] (sem) {语义分析};
  \node[data, right=2cm of sem] (ir) {IR 中间表示};
  \node[stage, below=of sem] (ver) {形式化验证引擎};
  \node[data, right=2cm of ver] (report) {验证报告};
  \node[stage, below=of ver] (gen) {代码生成};
  \node[data, right=2cm of gen] (rust) {Rust 状态机};

  \draw[arr] (src) -- (parse);
  \draw[arr] (parse) -- (sem);
  \draw[arr] (parse) -- (ast);
  \draw[arr] (sem) -- (ver);
  \draw[arr] (sem) -- (ir);
  \draw[arr] (ver) -- (gen);
  \draw[arr] (ver) -- (report);
  \draw[arr] (gen) -- (rust);
\end{tikzpicture}
\caption{RustPLC 编译流水线(含代码生成阶段)}
\label{fig:pipeline}
\end{figure}

\subsection{阶段一:解析(Parser)}

使用 \texttt{pest} PEG 解析器,将 \texttt{.plc} 源文件转换为抽象语法树(AST)。
语法文件 \texttt{plc.pest} 定义了 150 条规则,覆盖三大段落:

\begin{itemize}
  \item \texttt{[topology]}——设备声明与物理连接
  \item \texttt{[constraints]}——安全/时序/因果约束
  \item \texttt{[tasks]}——控制任务与步骤(状态机)
\end{itemize}

\subsection{阶段二:语义分析(Semantic Analysis)}

将 AST 降低为四种中间表示:

\begin{table}[htbp]
\centering
\begin{tabular}{lll}
\toprule
\textbf{IR 结构} & \textbf{数据结构} & \textbf{用途} \\
\midrule
拓扑图 & \texttt{petgraph::DiGraph} & 设备连接关系 \\
状态机 & 状态 + 转移 + 守卫 & 控制流建模 \\
约束集 & 安全/时序/因果规则 & 验证目标 \\
时序模型 & 动作$\to$时间区间映射 & 时序分析 \\
\bottomrule
\end{tabular}
\caption{四种中间表示}
\end{table}

\subsection{阶段三:形式化验证}

四个独立的验证引擎并行运行:

\begin{enumerate}
  \item \textbf{安全检查器}——有界模型检验(BMC)+ $k$-归纳,可选 Z3 SMT 求解器
  \item \textbf{活性检查器}——强连通分量(SCC)分析 + 死锁检测
  \item \textbf{时序检查器}——最坏情况关键路径分析
  \item \textbf{因果检查器}——拓扑图 BFS 可达性验证
\end{enumerate}

\subsection{阶段四:代码生成(v3 新增)}

验证通过后,编译器将 IR 转换为可执行的 Rust 代码。详见第~\ref{sec:codegen} 章。


%% ============================================================
\section{DSL 语言设计}

每个 \texttt{.plc} 文件由三个段落组成。下面以``传送带冲压系统''为例逐段讲解。

\subsection{拓扑段 \texttt{[topology]}}

声明物理设备及其连接关系:

\begin{lstlisting}[language=PLC]
[topology]

device Y0: digital_output
device Y1: digital_output
device X0: digital_input
device X1: digital_input
device X2: digital_input
device X3: digital_input

device start_button: digital_input {
    connected_to: X3
    debounce: 20ms
}

device conveyor_motor: motor {
    connected_to: Y0
    rated_speed: 30rpm
    ramp_time: 100ms
}

device stamp_valve: solenoid_valve {
    connected_to: Y1
    response_time: 15ms
}

device stamp_head: cylinder {
    connected_to: stamp_valve
    stroke_time: 250ms
    retract_time: 200ms
}

device sensor_in_position: sensor {
    type: proximity
    connected_to: X0
    detects: conveyor_motor.position_A
}

device sensor_stamp_down: sensor {
    type: magnetic
    connected_to: X1
    detects: stamp_head.extended
}

device sensor_stamp_up: sensor {
    type: magnetic
    connected_to: X2
    detects: stamp_head.retracted
}
\end{lstlisting}

\texttt{connected\_to} 表示上游连接:\texttt{stamp\_head} 连接到 \texttt{stamp\_valve},
意味着电磁阀驱动气缸。编译器据此构建有向拓扑图。

\subsection{约束段 \texttt{[constraints]}}

声明安全、时序和因果约束:

\begin{lstlisting}[language=PLC]
[constraints]

safety: stamp_head.extended conflicts_with conveyor_motor.on
    reason: "冲压头下压时传送带不能运行"

timing: task.cycle must_complete_within 3000ms
    reason: "单个冲压周期不应超过3秒"

causality: Y1 -> stamp_valve -> stamp_head -> sensor_stamp_down
causality: Y1 -> stamp_valve -> stamp_head -> sensor_stamp_up
\end{lstlisting}

\begin{itemize}
  \item \textbf{safety}:\texttt{conflicts\_with} 表示两个状态互斥,\texttt{requires} 表示蕴含。
  \item \textbf{timing}:\texttt{must\_complete\_within} 设定上界,\texttt{must\_start\_after} 设定下界。
  \item \textbf{causality}:用 \texttt{->} 链声明信号传播路径。
\end{itemize}

\subsection{任务段 \texttt{[tasks]}}

定义控制逻辑为状态机:

\begin{lstlisting}[language=PLC]
[tasks]

task cycle:
    step feed:
        action: set conveyor_motor on
        wait: sensor_in_position == true
        timeout: 1500ms -> goto fault_handler
    step stop_belt:
        action: set conveyor_motor off
    step press_down:
        action: extend stamp_head
        wait: sensor_stamp_down == true
        timeout: 500ms -> goto fault_handler
    step press_up:
        action: retract stamp_head
        wait: sensor_stamp_up == true
        timeout: 500ms -> goto fault_handler
    on_complete: goto ready

task fault_handler:
    step emergency:
        action: retract stamp_head
        action: set conveyor_motor off
    step report:
        action: log "冲压系统故障:动作超时"
    on_complete: goto ready

task ready:
    step wait_start:
        wait: start_button == true
        allow_indefinite_wait: true
    on_complete: goto cycle
\end{lstlisting}

%% ============================================================
\section{验证引擎详解}

\subsection{安全检查器:有界模型检验}

安全检查器的核心算法是\textbf{有界模型检验}(Bounded Model Checking, BMC),
辅以 $k$-\textbf{归纳}($k$-induction)来获得完备证明。

\subsubsection{状态空间建模}

系统的具体状态是一个元组:
\[
\sigma = (q, d_1, d_2, \ldots, d_n)
\]
其中 $q \in Q$ 是控制状态机的当前状态,$d_i \in D_i$ 是第 $i$ 个设备的当前状态。

\subsubsection{BMC 搜索}

从初始状态 $\sigma_0$ 出发,BFS 展开所有可达状态,直到深度 $k$:

\[
\text{Reach}_k = \{ \sigma \mid \exists \sigma_0 \xrightarrow{t_1} \sigma_1 \xrightarrow{t_2} \cdots \xrightarrow{t_k} \sigma, \; \sigma_0 \in \mathcal{I} \}
\]

对每个 \texttt{conflicts\_with} 约束 $\phi$,检查:
\[
\forall \sigma \in \text{Reach}_k : \sigma \models \neg\phi
\]

\subsubsection{$k$-归纳}

如果 BFS 在深度 $k$ 时状态空间已收敛(无新状态),则获得\textbf{完备证明}:
\[
\text{Reach}_k = \text{Reach}_{k+1} \implies \text{Reach}_k = \text{Reach}_\infty
\]

在我们的传送带冲压示例中,安全检查器在深度 8 获得完备证明,
证明了``冲压头下压时传送带不会运行''这一安全性质。

\subsection{活性检查器:死锁检测}

活性检查器执行四项检查:

\begin{enumerate}
  \item \textbf{无超时等待}:每个 \texttt{wait} 必须有 \texttt{timeout} 或 \texttt{allow\_indefinite\_wait}。
  \item \textbf{不可达声明验证}:标记为 \texttt{unreachable} 的 \texttt{on\_complete} 必须所有路径都跳走。
  \item \textbf{零出度检测}:非终态必须有出边。
  \item \textbf{SCC 分析}:使用 Kosaraju 算法检测可能困住执行的环路。
\end{enumerate}

\subsection{时序检查器:关键路径分析}

时序检查器沿拓扑连接链累加设备物理参数,计算最坏情况执行时间。

以冲压周期为例:

\begin{table}[htbp]
\centering
\begin{tabular}{llr}
\toprule
\textbf{步骤} & \textbf{关键动作} & \textbf{最坏时间} \\
\midrule
feed & 电机启动 + 等待到位 & 1500ms(超时上界) \\
stop\_belt & 电机停止 & 100ms \\
press\_down & 电磁阀响应 + 气缸行程 & 500ms(超时上界) \\
press\_up & 气缸回程 & 500ms(超时上界) \\
\midrule
\textbf{总计} & & \textbf{2600ms} \\
\bottomrule
\end{tabular}
\caption{冲压周期最坏情况时序分析}
\end{table}

$2600\text{ms} < 3000\text{ms}$,时序约束满足。

\subsection{因果检查器:信号可达性}

因果检查器在拓扑图上执行 BFS,验证声明的因果链中每一跳都有对应的物理连接:

\[
\text{Y1} \xrightarrow{\text{electrical}} \text{stamp\_valve} \xrightarrow{\text{pneumatic}} \text{stamp\_head} \xrightarrow{\text{logical}} \text{sensor\_stamp\_down}
\]

如果任何一跳断开(例如 \texttt{connected\_to} 配置错误),编译器会报告断裂位置并给出修复建议。


%% ============================================================
\section{完整示例:传送带冲压系统}

\subsection{场景描述}

一条自动化产线上,工件由传送带送入冲压工位。传感器检测到工件到位后,
传送带停止,冲压头下压完成冲压,然后回位,传送带继续运行。

关键安全约束:\textbf{冲压头下压时传送带绝对不能运行},否则工件移位会导致冲压偏移,
损坏模具甚至造成安全事故。

\subsection{系统拓扑}

\begin{figure}[htbp]
\centering
\begin{tikzpicture}[
  dev/.style={rectangle, draw, rounded corners, minimum width=2.2cm, minimum height=0.8cm, font=\small},
  io/.style={dev, fill=green!15},
  act/.style={dev, fill=orange!15},
  sen/.style={dev, fill=cyan!15},
  conn/.style={-{Stealth[length=2.5mm]}, thick},
  label/.style={font=\tiny, midway, above},
  node distance=1.5cm and 2cm
]
  \node[io] (y0) {Y0};
  \node[io, right=3cm of y0] (y1) {Y1};
  \node[act, below=of y0] (motor) {conveyor\_motor};
  \node[act, below=of y1] (valve) {stamp\_valve};
  \node[act, below=of valve] (head) {stamp\_head};
  \node[sen, below left=1.2cm and 0cm of motor] (spos) {sensor\_in\_position};
  \node[sen, below left=1.2cm and -0.5cm of head] (sdown) {sensor\_stamp\_down};
  \node[sen, below right=1.2cm and -0.5cm of head] (sup) {sensor\_stamp\_up};

  \draw[conn] (y0) -- node[label] {电气} (motor);
  \draw[conn] (y1) -- node[label] {电气} (valve);
  \draw[conn] (valve) -- node[label,right] {气动} (head);
  \draw[conn, dashed] (motor) -- node[label,left] {检测} (spos);
  \draw[conn, dashed] (head) -- node[label,left] {检测} (sdown);
  \draw[conn, dashed] (head) -- node[label,right] {检测} (sup);
\end{tikzpicture}
\caption{传送带冲压系统拓扑图}
\label{fig:topo}
\end{figure}

\subsection{状态机}

\begin{figure}[htbp]
\centering
\begin{tikzpicture}[
  state/.style={rectangle, draw, rounded corners, minimum width=2cm, minimum height=0.7cm,
                fill=blue!10, font=\small},
  arr/.style={-{Stealth[length=2.5mm]}, thick},
  timeout/.style={arr, dashed, red},
  node distance=1.5cm
]
  \node[state] (feed) {cycle.feed};
  \node[state, below=of feed] (stop) {cycle.stop\_belt};
  \node[state, below=of stop] (down) {cycle.press\_down};
  \node[state, below=of down] (up) {cycle.press\_up};
  \node[state, right=3cm of feed] (fault) {fault\_handler};
  \node[state, below=3cm of fault] (ready) {ready.wait\_start};

  \draw[arr] (feed) -- node[left,font=\tiny] {到位} (stop);
  \draw[arr] (stop) -- node[left,font=\tiny] {always} (down);
  \draw[arr] (down) -- node[left,font=\tiny] {下压到位} (up);
  \draw[arr] (up) -- node[left,font=\tiny] {回位到位} (ready);
  \draw[timeout] (feed) -- node[above,font=\tiny] {1500ms} (fault);
  \draw[timeout] (down.east) -- (fault);
  \draw[timeout] (up.east) -- (fault);
  \draw[arr] (fault) -- (ready);
  \draw[arr] (ready.west) -- +(-1.5,0) |- (feed.west);
\end{tikzpicture}
\caption{传送带冲压系统状态机(红色虚线为超时跳转)}
\label{fig:sm}
\end{figure}

\subsection{运行验证}

在项目根目录执行:

\begin{Verbatim}[frame=single,fontsize=\small]
$ cargo run -- examples/industrial/conveyor_stamp.plc
\end{Verbatim}

输出:

\begin{Verbatim}[frame=single,fontsize=\small]
验证通过:
  - Safety: 完备证明(深度 8)
  - Liveness: 通过
  - Timing: 通过
  - Causality: 通过
\end{Verbatim}

四项验证全部通过,含义如下:

\begin{itemize}
  \item \textbf{Safety 完备证明}:在深度 8 的状态空间搜索中,状态已收敛。
        数学上证明了在\textbf{任意}执行路径上,冲压头下压时传送带都不会运行。
  \item \textbf{Liveness 通过}:所有等待都有超时保护或明确的无限等待许可,
        不存在死锁。
  \item \textbf{Timing 通过}:最坏情况关键路径 2600ms $<$ 约束上界 3000ms。
  \item \textbf{Causality 通过}:从 Y1 到 sensor\_stamp\_down 和 sensor\_stamp\_up
        的信号传播路径在拓扑图中完整连通。
\end{itemize}

%% ============================================================
\section{深度追踪:一条安全约束如何变成数学证明}

前面我们从宏观上介绍了编译器架构和验证引擎。但``自动化验证''到底是怎么实现的?
本节以传送带冲压系统中的一条安全约束为例,逐函数、逐数据结构地追踪它从 DSL 文本
变成数学证明的完整路径。

我们追踪的目标是这一行:

\begin{Verbatim}[frame=single,fontsize=\small]
safety: stamp_head.extended conflicts_with conveyor_motor.on
\end{Verbatim}

整个过程经历 5 个阶段,涉及 12 个关键函数和 8 个数据结构。

\subsection{阶段一:文本 $\to$ AST(解析)}

\texttt{pest} 解析器用 \texttt{plc.pest} 中的语法规则匹配文本。
\texttt{parse\_safety\_constraint()} 遍历语法树,产出 AST 节点:

\begin{Verbatim}[frame=single,fontsize=\small]
SafetyConstraint {
    line: 47,
    left:  StateReference { device: "stamp_head",     state: "extended" },
    relation: ConflictsWith,
    right: StateReference { device: "conveyor_motor", state: "on" },
    reason: Some("冲压头下压时传送带不能运行...")
}
\end{Verbatim}

\subsection{阶段二:AST $\to$ IR(语义分析)}

\texttt{build\_constraint\_set\_from\_ast()} 对每条 safety 约束执行两步:
验证引用合法性(设备是否存在、状态是否合法),然后降低为 IR:

\begin{Verbatim}[frame=single,fontsize=\small]
SafetyRule {
    left:     StateExpr { device: "stamp_head",     state: "extended" },
    relation: ConflictsWith,
    right:    StateExpr { device: "conveyor_motor", state: "on" },
}
\end{Verbatim}

\subsection{阶段三:IR $\to$ SafetyModel(模型构建)}

\texttt{SafetyModel::from\_inputs()} 构建验证模型:
\begin{enumerate}
  \item \texttt{collect\_device\_domains()} 为每个设备建立有限状态域和数值索引
  \item \texttt{transition\_effects()} 将状态机转移映射为设备状态变更
  \item \texttt{scc\_minimum\_depth()} 用 Kosaraju 算法确定搜索深度(本例为 8)
\end{enumerate}

\subsection{阶段四:BFS 穷举搜索}

\texttt{analyze\_rule()} 从初始状态 BFS 展开,对每个可达状态检查冲突:

\begin{table}[htbp]
\centering
\small
\begin{tabular}{clccc}
\toprule
\textbf{深度} & \textbf{控制状态} & \textbf{stamp\_head} & \textbf{conveyor\_motor} & \textbf{冲突?} \\
\midrule
0 & cycle.feed & retracted & off & No \\
1 & cycle.stop\_belt & retracted & on & No \\
2 & cycle.press\_down & retracted & off & No \\
3 & cycle.press\_up & extended & off & No \\
4 & ready.wait\_start & retracted & off & No \\
$\vdots$ & $\vdots$ & $\vdots$ & $\vdots$ & $\vdots$ \\
\bottomrule
\end{tabular}
\caption{BFS 搜索过程(关键设备状态追踪)}
\end{table}

深度 3:\texttt{stamp\_head} 变为 extended 时,\texttt{conveyor\_motor} 已经是 off。
BFS 在深度 8 穷尽所有可达状态,未发现冲突。

\subsection{阶段五:证明判定}

$\text{Reach}_8 = \text{Reach}_9$,状态空间收敛,获得完备证明:
\[
\forall \sigma \in \text{Reach}_\infty : \neg(\text{stamp\_head.extended} \wedge \text{conveyor\_motor.on})
\]

\begin{figure}[htbp]
\centering
\begin{tikzpicture}[
  box/.style={rectangle, draw, rounded corners, minimum width=4cm, minimum height=0.8cm,
              font=\small, fill=blue!5},
  func/.style={rectangle, draw, minimum width=3.5cm, minimum height=0.6cm,
               font=\scriptsize\ttfamily, fill=green!8},
  data/.style={rectangle, draw, dashed, minimum width=3.5cm, minimum height=0.6cm,
               font=\scriptsize, fill=yellow!8},
  arr/.style={-{Stealth[length=2.5mm]}, thick},
  node distance=0.8cm
]
  \node[data] (src) {DSL: safety: stamp\_head.extended ...};
  \node[func, below=of src] (f1) {parse\_safety\_constraint()};
  \node[data, below=of f1] (ast) {AST::SafetyConstraint};
  \node[func, below=of ast] (f2) {build\_constraint\_set\_from\_ast()};
  \node[data, below=of f2] (ir) {IR::SafetyRule};
  \node[func, below=of ir] (f3) {SafetyModel::from\_inputs()};
  \node[data, below=of f3] (model) {SafetyModel + RuleBinding};
  \node[func, below=of model] (f4) {analyze\_rule() --- BFS};
  \node[data, below=of f4] (outcome) {SearchOutcome: 无反例 + 已收敛};
  \node[box, below=of outcome] (proof) {完备证明 (Complete)};

  \draw[arr] (src) -- (f1);
  \draw[arr] (f1) -- (ast);
  \draw[arr] (ast) -- (f2);
  \draw[arr] (f2) -- (ir);
  \draw[arr] (ir) -- (f3);
  \draw[arr] (f3) -- (model);
  \draw[arr] (model) -- (f4);
  \draw[arr] (f4) -- (outcome);
  \draw[arr] (outcome) -- (proof);
\end{tikzpicture}
\caption{安全约束从文本到证明的完整数据流}
\label{fig:dataflow}
\end{figure}


%% ============================================================
%% 第二部分:执行域 (BC2) —— v3 新增
%% ============================================================

\part{执行域 (BC2):从 IR 到确定性执行}

\section{代码生成:IR $\to$ Rust 状态机}
\label{sec:codegen}

验证通过后,编译器将 IR 转换为可独立编译运行的 Rust 项目。
代码生成器的设计原则是:\textbf{生成的代码是纯粹的状态机,没有 I/O 副作用}。
所有 I/O 通过 \texttt{HalBackend} trait 完成,使得同一份生成代码可以驱动
仿真后端、Modbus 后端或 FPGA 后端。

\subsection{生成产物}

代码生成器输出一个完整的 Cargo workspace:

\begin{Verbatim}[frame=single,fontsize=\small]
generated/
├── Cargo.toml          # workspace 根
├── src/
│   ├── main.rs         # 入口:创建 HAL + 启动 ScanCycleEngine
│   ├── plc_state.rs    # PlcState enum + scan_cycle 函数
│   └── device_map.rs   # 设备名 → HAL 地址映射
└── config/
    └── hal.toml        # HAL 后端配置
\end{Verbatim}

\subsection{PlcState enum}

每个控制状态机的状态对应一个 enum variant:

\begin{lstlisting}[language=RustPLC]
#[derive(Debug, Clone, Copy, PartialEq, Eq)]
pub enum PlcState {
    CycleFeed,
    CycleStopBelt,
    CyclePressDown,
    CyclePressUp,
    FaultHandlerEmergency,
    FaultHandlerReport,
    ReadyWaitStart,
}
\end{lstlisting}

命名规则:\texttt{task\_name} + \texttt{step\_name},驼峰化。
编译器保证 enum 的 variant 集合与 IR 中的状态集合一一对应。

\subsection{scan\_cycle 函数}

核心生成产物——一个纯粹的 \texttt{match} 状态机:

\begin{lstlisting}[language=RustPLC]
pub fn scan_cycle(
    state: &mut PlcState,
    hal: &mut impl HalBackend,
    timers: &mut TimerBank,
) {
    match state {
        PlcState::CycleFeed => {
            hal.write_digital_output("conveyor_motor", true);
            timers.start(0, 1500); // timer 0: feed timeout
            if hal.read_digital_input("sensor_in_position") {
                timers.stop(0);
                *state = PlcState::CycleStopBelt;
            } else if timers.expired(0) {
                *state = PlcState::FaultHandlerEmergency;
            }
        }
        PlcState::CycleStopBelt => {
            hal.write_digital_output("conveyor_motor", false);
            *state = PlcState::CyclePressDown;
        }
        PlcState::CyclePressDown => {
            hal.write_digital_output("stamp_valve", true);
            timers.start(1, 500);
            if hal.read_digital_input("sensor_stamp_down") {
                timers.stop(1);
                *state = PlcState::CyclePressUp;
            } else if timers.expired(1) {
                *state = PlcState::FaultHandlerEmergency;
            }
        }
        // ... 其余状态类似
    }
}
\end{lstlisting}

关键设计决策:

\begin{itemize}
  \item \textbf{无 I/O 副作用}:\texttt{scan\_cycle} 只通过 \texttt{hal} 参数读写 I/O,
        不直接访问硬件。这使得单元测试可以用 \texttt{SimBackend} 驱动。
  \item \textbf{定时器外置}:\texttt{TimerBank} 由引擎管理,\texttt{scan\_cycle} 只负责
        启动/停止/查询。这避免了生成代码中出现时间相关的副作用。
  \item \textbf{单次扫描语义}:每次调用 \texttt{scan\_cycle} 只执行一次状态转移判断,
        不会在内部循环。这与传统 PLC 的扫描周期语义一致。
\end{itemize}

\section{扫描周期引擎}

\texttt{ScanCycleEngine} 是 BC2 的核心组件,负责以固定周期驱动状态机:

\begin{lstlisting}[language=RustPLC]
pub struct ScanCycleEngine<H: HalBackend> {
    state: PlcState,
    hal: H,
    timers: TimerBank,
    cycle_ms: u64,
}

impl<H: HalBackend> ScanCycleEngine<H> {
    pub fn run_cycle(&mut self) -> Result<(), HalError> {
        // 1. 读取所有输入
        self.hal.refresh_inputs()?;

        // 2. 执行状态机
        scan_cycle(&mut self.state, &mut self.hal, &mut self.timers);

        // 3. 写出所有输出
        self.hal.flush_outputs()?;

        // 4. 推进定时器
        self.timers.tick(self.cycle_ms);

        Ok(())
    }
}
\end{lstlisting}

四步扫描周期的顺序是严格的:

\begin{enumerate}
  \item \texttt{refresh\_inputs()}——批量读取所有输入信号,建立本周期的一致性快照。
  \item \texttt{scan\_cycle()}——基于快照执行状态转移,产生输出指令。
  \item \texttt{flush\_outputs()}——批量写出所有输出信号。
  \item \texttt{timers.tick()}——推进所有活跃定时器。
\end{enumerate}

这个顺序保证了:在一个扫描周期内,状态机看到的输入是一致的(不会在执行过程中变化),
输出也是原子性的(不会出现半写状态)。这与 IEC 61131-3 标准的扫描周期语义一致。

\section{定时器组}

\texttt{TimerBank} 管理 32 个独立的倒计时定时器:

\begin{lstlisting}[language=RustPLC]
pub struct TimerBank {
    slots: [TimerSlot; 32],
}

pub struct TimerSlot {
    active: bool,
    remaining_ms: u64,
}

impl TimerBank {
    pub fn start(&mut self, id: usize, duration_ms: u64);
    pub fn stop(&mut self, id: usize);
    pub fn expired(&self, id: usize) -> bool;
    pub fn tick(&mut self, elapsed_ms: u64);
}
\end{lstlisting}

代码生成器为每个 \texttt{timeout} 分配一个定时器槽位。
\texttt{tick()} 在每个扫描周期末尾调用,将所有活跃定时器的剩余时间减去周期时长。


%% ============================================================
%% 第三部分:硬件适配域 (BC3)
%% ============================================================

\part{硬件适配域 (BC3):HalBackend 与三种模式}

\section{HalBackend trait 设计}

整个平台的核心解耦点是 \texttt{HalBackend} trait。它定义了运行时与物理世界之间的契约:

\begin{lstlisting}[language=RustPLC]
pub trait HalBackend {
    /// 读取数字输入信号
    fn read_digital_input(&self, device: &str) -> bool;

    /// 写入数字输出信号
    fn write_digital_output(&mut self, device: &str, value: bool);

    /// 批量刷新所有输入(从物理/仿真世界读取)
    fn refresh_inputs(&mut self) -> Result<(), HalError>;

    /// 批量写出所有输出(到物理/仿真世界)
    fn flush_outputs(&mut self) -> Result<(), HalError>;
}
\end{lstlisting}

这个 trait 的设计遵循\textbf{依赖倒置原则}(Dependency Inversion Principle):
高层模块(\texttt{ScanCycleEngine})不依赖低层模块(Modbus/FPGA/内存),
两者都依赖抽象(\texttt{HalBackend})。

\subsection{为什么是 \texttt{\&str} 而不是地址?}

\texttt{read\_digital\_input} 的参数是设备名(\texttt{\&str}),而不是 Modbus 地址或 GPIO 编号。
这是有意为之的设计:

\begin{itemize}
  \item 生成的 \texttt{scan\_cycle} 代码使用设备名,与 DSL 中的名称一致,可读性好。
  \item 设备名到物理地址的映射由 \texttt{DeviceMapping} 配置文件完成,不硬编码在生成代码中。
  \item 同一份生成代码可以通过不同的配置文件驱动不同的 HAL 后端。
\end{itemize}

\subsection{DeviceMapping 配置}

设备名到物理地址的映射通过 TOML 配置文件完成:

\begin{Verbatim}[frame=single,fontsize=\small]
# config/hal_modbus.toml
[backend]
type = "modbus_tcp"
host = "192.168.1.100"
port = 502
unit_id = 1

[mapping]
conveyor_motor = { type = "coil", address = 0 }
stamp_valve    = { type = "coil", address = 1 }
sensor_in_position = { type = "discrete_input", address = 0 }
sensor_stamp_down  = { type = "discrete_input", address = 1 }
sensor_stamp_up    = { type = "discrete_input", address = 2 }
start_button       = { type = "discrete_input", address = 3 }
\end{Verbatim}

\section{模式 A:SimBackend}

\texttt{SimBackend} 是最简单的 HAL 实现,用于 CI 测试和单元测试:

\begin{lstlisting}[language=RustPLC]
pub struct SimBackend {
    inputs: HashMap<String, bool>,
    outputs: HashMap<String, bool>,
}

impl HalBackend for SimBackend {
    fn read_digital_input(&self, device: &str) -> bool {
        *self.inputs.get(device).unwrap_or(&false)
    }

    fn write_digital_output(&mut self, device: &str, value: bool) {
        self.outputs.insert(device.to_string(), value);
    }

    fn refresh_inputs(&mut self) -> Result<(), HalError> {
        Ok(()) // 内存后端无需刷新
    }

    fn flush_outputs(&mut self) -> Result<(), HalError> {
        Ok(()) // 内存后端无需刷新
    }
}
\end{lstlisting}

测试代码可以直接操作 \texttt{inputs} HashMap 来模拟传感器信号:

\begin{lstlisting}[language=RustPLC]
#[test]
fn test_conveyor_stamp_cycle() {
    let mut engine = ScanCycleEngine::new(
        PlcState::ReadyWaitStart,
        SimBackend::new(),
        50, // 50ms cycle
    );

    // 模拟按下启动按钮
    engine.hal.inputs.insert("start_button".into(), true);
    engine.run_cycle().unwrap();
    assert_eq!(engine.state, PlcState::CycleFeed);

    // 模拟工件到位
    engine.hal.inputs.insert("sensor_in_position".into(), true);
    engine.run_cycle().unwrap();
    assert_eq!(engine.state, PlcState::CycleStopBelt);
}
\end{lstlisting}

\section{模式 B:ModbusBackend(计划中)}

\texttt{ModbusBackend} 使用 \texttt{tokio-modbus} 库通过 Modbus RTU/TCP 协议
与远程 I/O 从站通信:

\begin{lstlisting}[language=RustPLC]
pub struct ModbusBackend {
    ctx: tokio_modbus::client::Context,
    mapping: DeviceMapping,
    input_cache: HashMap<String, bool>,
    output_buffer: HashMap<String, bool>,
}

impl HalBackend for ModbusBackend {
    fn refresh_inputs(&mut self) -> Result<(), HalError> {
        // 批量读取所有 discrete_input 寄存器
        let addrs = self.mapping.discrete_input_range();
        let values = self.ctx.read_discrete_inputs(addrs)?;
        for (name, addr) in self.mapping.inputs() {
            self.input_cache.insert(name, values[addr]);
        }
        Ok(())
    }

    fn flush_outputs(&mut self) -> Result<(), HalError> {
        // 批量写入所有 coil 寄存器
        let coils: Vec<bool> = self.mapping.coil_range()
            .map(|addr| self.output_buffer.get(&addr).copied()
                 .unwrap_or(false))
            .collect();
        self.ctx.write_multiple_coils(0, &coils)?;
        Ok(())
    }
    // ...
}
\end{lstlisting}

关键设计:\texttt{refresh\_inputs()} 和 \texttt{flush\_outputs()} 各只发一次 Modbus 请求,
而不是每个设备一次。这将 Modbus 通信开销从 $O(n)$ 降到 $O(1)$,
对于 50ms 扫描周期至关重要。

\section{模式 C:FpgaBackend(计划中)}

\texttt{FpgaBackend} 通过 UART 与 iCESugar-pro FPGA 通信,
FPGA 内部运行硬件 Modbus RTU slave 状态机:

\begin{Verbatim}[frame=single,fontsize=\small]
ARM 主控 (Linux)                    iCESugar-pro FPGA
┌──────────────┐                   ┌──────────────────┐
│ FpgaBackend  │  UART 115200bps   │ Modbus RTU slave │
│              │ ──── TX ────────→ │   (Verilog)      │
│              │ ←─── RX ──────── │                  │
│              │                   │ PMOD GPIO        │
│              │                   │  → 继电器/LED    │
│              │                   │  ← 按钮/传感器   │
└──────────────┘                   └──────────────────┘
\end{Verbatim}

FPGA 的优势是\textbf{确定性时序}:GPIO 翻转在纳秒级完成,
不受操作系统调度影响。这对于需要精确时序的安全关键应用至关重要。


%% ============================================================
%% 第四部分:DDD 架构全局视角
%% ============================================================

\part{DDD 架构:限界上下文与集成模式}

\section{为什么用 DDD}

RustPLC 不是一个单体应用,而是一个跨越编译器、运行时、通信协议、电路仿真、
力学仿真和 FPGA 硬件的异构系统。传统的分层架构(Layer Architecture)
无法表达这些子系统之间的复杂关系——它们不是简单的上下层调用,
而是通过不同的协议和数据格式进行双向通信。

Domain-Driven Design(DDD)的\textbf{限界上下文}(Bounded Context)
提供了更好的组织方式:每个上下文有明确的职责边界、领域语言和集成契约。

\section{五个限界上下文}

\begin{figure}[htbp]
\centering
\begin{tikzpicture}[
  bc/.style={rectangle, draw, thick, rounded corners=5pt, minimum width=4.5cm,
             minimum height=2cm, font=\small, align=center},
  arr/.style={-{Stealth[length=3mm]}, thick},
  darr/.style={-{Stealth[length=3mm]}, thick, dashed},
  label/.style={font=\tiny, fill=white, inner sep=1pt},
  node distance=2cm and 3cm
]
  \node[bc, fill=blue!10] (bc1) {BC1: 验证域\\{\scriptsize .plc → AST → IR → 证明}\\{\scriptsize crate: rustplc\_compiler}};
  \node[bc, fill=green!10, below=of bc1] (bc2) {BC2: 执行域\\{\scriptsize IR → 状态机 + 扫描引擎}\\{\scriptsize crate: rustplc\_runtime}};
  \node[bc, fill=orange!10, below=of bc2] (bc3) {BC3: 硬件适配域\\{\scriptsize HalBackend trait}\\{\scriptsize crate: rustplc\_hal/modbus/fpga}};
  \node[bc, fill=purple!10, below left=2cm and -1cm of bc3] (bc4) {BC4: 仿真域\\{\scriptsize 虚拟 I/O + Gnucap + jtufem}\\{\scriptsize extern: esxi-arm/gnucap/jtufem}};
  \node[bc, fill=red!10, below right=2cm and -1cm of bc3] (bc5) {BC5: 硬件域\\{\scriptsize FPGA + GPIO}\\{\scriptsize extern: iCESugar-pro}};

  \draw[arr] (bc1) -- node[label] {IR (Shared Kernel)} (bc2);
  \draw[arr] (bc2) -- node[label] {HalBackend (DI)} (bc3);
  \draw[arr] (bc3) -- node[label, left] {Modbus 帧} (bc4);
  \draw[arr] (bc3) -- node[label, right] {Modbus RTU} (bc5);
  \draw[darr] (bc4) -- node[label, left] {DI 回写} (bc3);
  \draw[darr] (bc5) -- node[label, right] {GPIO 回读} (bc3);
\end{tikzpicture}
\caption{五个限界上下文及其集成关系(虚线为闭环回路)}
\label{fig:ddd}
\end{figure}

\section{集成模式}

上下文之间的集成不是随意的——每种集成关系都对应 DDD 中的一种标准模式:

\begin{table}[htbp]
\centering
\small
\begin{tabular}{llp{6cm}}
\toprule
\textbf{关系} & \textbf{模式} & \textbf{说明} \\
\midrule
BC1 → BC2 & 共享内核 & IR 数据结构(\texttt{StateMachine}, \texttt{TopologyGraph})
                        由两个上下文共同拥有,修改需要双方同意 \\
BC2 → BC3 & 依赖倒置 & \texttt{HalBackend} trait 定义在 BC3,
                        但由 BC2 的 \texttt{ScanCycleEngine} 消费 \\
BC3 → BC4 & 发布语言 & Modbus RTU/TCP 是公开标准协议,
                        BC3 不需要了解 BC4 的内部实现 \\
BC3 → BC5 & 发布语言 & 同上,Modbus RTU 帧格式是契约 \\
BC4 → BC3 & 遵从者 & BC4 必须按照 Modbus 寄存器映射表回写数据,
                      不能自定义协议 \\
BC5 → BC3 & 遵从者 & 同上,FPGA 必须实现标准 Modbus RTU slave \\
\bottomrule
\end{tabular}
\caption{限界上下文集成模式}
\end{table}

\section{Crate 映射}

每个限界上下文对应一个或多个 Rust crate:

\begin{table}[htbp]
\centering
\small
\begin{tabular}{llp{7cm}}
\toprule
\textbf{BC} & \textbf{Crate} & \textbf{职责} \\
\midrule
BC1 & \texttt{rustplc\_compiler} & 解析、语义分析、四大验证引擎、代码生成 \\
BC2 & \texttt{rustplc\_runtime} & \texttt{ScanCycleEngine}、\texttt{TimerBank}、
                                   生成代码的运行时支撑 \\
BC3 & \texttt{rustplc\_hal} & \texttt{HalBackend} trait、\texttt{SimBackend}、
                               \texttt{DeviceMapping} \\
BC3 & \texttt{rustplc\_modbus} & \texttt{ModbusBackend}(tokio-modbus) \\
BC3/BC5 & \texttt{rustplc\_fpga} & \texttt{FpgaBackend}(UART Modbus RTU) \\
跨 BC & \texttt{rustplc\_orchestrator} & 配置驱动的模式选择、启动编排 \\
\bottomrule
\end{tabular}
\caption{限界上下文到 Crate 的映射}
\end{table}

%% ============================================================
%% 第五部分:仿真域 (BC4)
%% ============================================================

\part{仿真域 (BC4):虚拟工厂的闭环数据流}

\section{虚拟工厂架构}

模式 B 的核心价值是:\textbf{不需要任何实体硬件,纯软件构建完整的控制闭环}。
这使得开发者可以在笔记本电脑上验证控制逻辑的正确性,包括物理延迟和力学响应。

\begin{figure}[htbp]
\centering
\begin{tikzpicture}[
  vm/.style={rectangle, draw, thick, rounded corners, minimum width=5cm,
             minimum height=1.5cm, font=\small, align=center},
  proc/.style={rectangle, draw, minimum width=3.5cm, minimum height=0.7cm,
               font=\scriptsize, fill=green!10},
  sim/.style={rectangle, draw, dashed, minimum width=3.5cm, minimum height=0.7cm,
              font=\scriptsize, fill=yellow!10},
  arr/.style={-{Stealth[length=3mm]}, thick},
  darr/.style={-{Stealth[length=3mm]}, thick, dashed, blue},
  node distance=1cm
]
  \node[vm, fill=blue!5] (vm1) {VM1: PLC 主控\\{\tiny Cortex-A53 + PREEMPT\_RT}};
  \node[proc, below=0.5cm of vm1] (runtime) {ScanCycleEngine<ModbusBackend>};
  \node[vm, fill=orange!5, below=1.5cm of runtime] (vm2) {VM2: 虚拟 I/O 从站\\{\tiny Modbus slave daemon}};
  \node[sim, below left=0.8cm and -1cm of vm2] (gnucap) {188-Gnucap 电路仿真};
  \node[sim, below right=0.8cm and -1cm of vm2] (jtufem) {jtufem-rs FEM 仿真};

  \draw[arr] (runtime) -- node[right, font=\tiny] {write\_coils} (vm2);
  \draw[darr] (vm2) -- node[left, font=\tiny] {read\_discrete\_inputs} (runtime);
  \draw[arr] (vm2) -- (gnucap);
  \draw[arr] (vm2) -- (jtufem);
  \draw[darr] (gnucap) -- node[left, font=\tiny] {DI 延迟注入} (vm2);
  \draw[darr] (jtufem) -- node[right, font=\tiny] {物理参数修正} (vm2);
\end{tikzpicture}
\caption{模式 B 虚拟工厂数据流(蓝色虚线为闭环回路)}
\label{fig:virtual-factory}
\end{figure}

\section{Modbus slave daemon}

虚拟 I/O 从站是一个 Modbus TCP slave 进程,维护两组寄存器:

\begin{itemize}
  \item \texttt{coils[]}——主控写入的输出信号(如电磁阀开关指令)
  \item \texttt{discrete\_inputs[]}——从站回写的输入信号(如传感器状态)
\end{itemize}

当主控写入一个 coil 时,从站不是简单地将对应的 DI 置位,
而是将写入事件转发给仿真引擎,由仿真引擎计算物理响应后再更新 DI。

\section{电路仿真集成(188-Gnucap)}

Gnucap 是一个开源 SPICE 电路仿真器。在虚拟工厂中,它负责模拟电气层面的物理延迟:

\subsection{线圈吸合延迟}

当主控写入 \texttt{stamp\_valve = ON} 时,电磁阀线圈不会瞬间吸合。
Gnucap 模拟 RL 电路的暂态响应:

\[
i(t) = \frac{V}{R}\left(1 - e^{-\frac{R}{L}t}\right)
\]

当电流达到吸合阈值 $I_{\text{pull-in}}$ 时,阀芯动作。
典型的 24V DC 电磁阀吸合延迟约 15ms。

\subsection{传感器信号滤波}

接近传感器的输出经过 RC 低通滤波器,信号上升沿有延迟:

\[
v_{\text{out}}(t) = V_{\text{cc}}\left(1 - e^{-\frac{t}{RC}}\right)
\]

当 $v_{\text{out}}$ 超过逻辑高电平阈值时,DI 才变为 true。
典型延迟约 5ms。

\subsection{仿真流程}

\begin{enumerate}
  \item 主控写 coil → 从站接收 Modbus 帧
  \item 从站将 coil 变更事件发送给 Gnucap
  \item Gnucap 运行瞬态仿真,计算延迟时间 $\Delta t$
  \item 从站启动定时器,$\Delta t$ 后更新对应的 DI
  \item 主控在下一个扫描周期读到更新后的 DI
\end{enumerate}

\section{力学仿真集成(jtufem-rs)}

jtufem-rs 是一个 Rust 实现的有限元方法(FEM)求解器。
在虚拟工厂中,它负责模拟机械层面的物理响应:

\subsection{气缸活塞杆应力分析}

当气缸伸出推动工件时,活塞杆承受轴向压力。FEM 计算:

\begin{itemize}
  \item 活塞杆应力分布(von Mises 应力)
  \item 最大变形量(影响冲压精度)
  \item 安全系数(是否超过材料屈服强度)
\end{itemize}

\subsection{疲劳寿命预测}

基于 S-N 曲线和 Miner 累积损伤理论,预测气缸的疲劳寿命:

\[
D = \sum_{i=1}^{k} \frac{n_i}{N_i}
\]

当累积损伤 $D \geq 1$ 时,预测疲劳失效。这为预测性维护提供了数据支撑。

\subsection{物理参数修正}

FEM 仿真的结果可以修正虚拟从站的行为参数:

\begin{itemize}
  \item 气缸行程时间:考虑负载后,实际行程时间可能比空载时长 10--20\%
  \item 定位精度:活塞杆变形导致工件位置偏移,影响传感器触发时机
  \item 振动特性:支架共振可能导致传感器误触发
\end{itemize}

%% ============================================================
%% 第六部分:硬件域 (BC5)
%% ============================================================

\part{硬件域 (BC5):FPGA 确定性 I/O}

\section{为什么需要 FPGA}

Linux 即使使用 PREEMPT\_RT 补丁,GPIO 翻转的抖动仍在 5--15$\mu$s 级别。
对于大多数工业控制场景这已经足够,但某些安全关键应用需要更高的确定性:

\begin{itemize}
  \item 紧急停机信号必须在微秒级响应
  \item 高速计数器(编码器脉冲)不能丢脉冲
  \item PWM 输出需要纳秒级精度
\end{itemize}

FPGA 的硬件并行性和确定性时序完美解决了这些问题。

\section{iCESugar-pro 开发板}

iCESugar-pro v1.3 基于 Lattice ECP5 FPGA,具有以下特性:

\begin{table}[htbp]
\centering
\begin{tabular}{ll}
\toprule
\textbf{参数} & \textbf{规格} \\
\midrule
FPGA 芯片 & Lattice ECP5 LFE5U-25F \\
逻辑单元 & 24K LUT \\
BRAM & 1008 Kbit \\
PMOD 接口 & 4 个(32 GPIO) \\
USB & USB-C(供电 + JTAG + UART) \\
工具链 & Yosys + nextpnr(开源) \\
价格 & \textasciitilde\$50 \\
\bottomrule
\end{tabular}
\caption{iCESugar-pro v1.3 规格}
\end{table}

\section{FPGA 内部架构}

FPGA 内部实现一个硬件 Modbus RTU slave 状态机:

\begin{figure}[htbp]
\centering
\begin{tikzpicture}[
  mod/.style={rectangle, draw, rounded corners, minimum width=2.5cm, minimum height=0.8cm,
              font=\small, fill=blue!10},
  io/.style={rectangle, draw, minimum width=2cm, minimum height=0.6cm,
             font=\small, fill=green!10},
  arr/.style={-{Stealth[length=2.5mm]}, thick},
  node distance=1cm and 2cm
]
  \node[mod] (uart) {UART 收发器};
  \node[mod, below=of uart] (modbus) {Modbus RTU\\状态机};
  \node[mod, below left=1cm and 0cm of modbus] (coils) {Coil 寄存器};
  \node[mod, below right=1cm and 0cm of modbus] (di) {DI 寄存器};
  \node[io, below=of coils] (out) {PMOD GPIO\\输出驱动};
  \node[io, below=of di] (in) {PMOD GPIO\\输入采样};

  \draw[arr] (uart) -- (modbus);
  \draw[arr] (modbus) -- (coils);
  \draw[arr] (modbus) -- (di);
  \draw[arr] (coils) -- (out);
  \draw[arr] (in) -- (di);
  \draw[arr] (di) -- (modbus);
  \draw[arr] (modbus) -- (uart);
\end{tikzpicture}
\caption{FPGA 内部 Modbus RTU slave 架构}
\label{fig:fpga}
\end{figure}

所有模块在硬件上并行运行:UART 收发、Modbus 帧解析、GPIO 驱动同时进行,
没有操作系统调度开销。GPIO 翻转延迟 $<$ 10ns。

\section{PMOD 外设连接}

通过 PMOD 接口连接工业级外设:

\begin{table}[htbp]
\centering
\begin{tabular}{lll}
\toprule
\textbf{PMOD 端口} & \textbf{外设} & \textbf{用途} \\
\midrule
PMOD-A[0:3] & 4 路继电器模块 & 数字输出(电磁阀/电机) \\
PMOD-B[0:3] & 4 路光耦隔离输入 & 数字输入(传感器/按钮) \\
PMOD-C[0:1] & UART 转 RS-485 & Modbus RTU 通信 \\
PMOD-D & 预留 & 扩展 \\
\bottomrule
\end{tabular}
\caption{PMOD 外设分配}
\end{table}

\section{虚实混合模式}

模式 C 的独特之处是可以\textbf{混合}虚拟和实体 I/O:

\begin{itemize}
  \item 部分设备连接到 FPGA GPIO(真实继电器、真实传感器)
  \item 部分设备连接到虚拟从站(Gnucap 仿真、jtufem 仿真)
  \item 主控的 \texttt{ModbusBackend} 同时与两个从站通信
\end{itemize}

这使得开发者可以逐步将虚拟设备替换为实体设备,实现渐进式部署。


%% ============================================================
%% 第七部分:总结与展望
%% ============================================================

\part{总结与展望}

\section{已完成的工作}

截至 2026 年 2 月,RustPLC 已完成以下里程碑:

\begin{table}[htbp]
\centering
\small
\begin{tabular}{llll}
\toprule
\textbf{Phase} & \textbf{限界上下文} & \textbf{核心交付物} & \textbf{状态} \\
\midrule
Phase 1 & BC1: 验证域 & DSL 解析器 + 四大验证引擎 & 已完成 \\
Phase 2 & BC2: 执行域 & 代码生成 + ScanCycleEngine + TimerBank & 已完成 \\
Phase 3 & BC3: 硬件适配域 & HalBackend trait + SimBackend + 配置 & 进行中 \\
\bottomrule
\end{tabular}
\caption{项目进度总览}
\end{table}

\subsection{验证能力}

编译器已通过 59 个测试(48 单元测试 + 5 集成测试 + 6 端到端验证测试),
覆盖以下验证场景:

\begin{itemize}
  \item 安全性:双缸互斥(通过/失败)、冲压安全(通过)
  \item 活性:无超时等待检测(失败)、正常流程(通过)
  \item 时序:约束过紧检测(失败)、正常时序(通过)
  \item 因果:链路断裂检测(失败)、完整链路(通过)
\end{itemize}

\subsection{代码生成能力}

代码生成器已能将验证通过的 \texttt{.plc} 文件转换为完整的 Rust Cargo 项目,
包含 \texttt{PlcState} enum、\texttt{scan\_cycle} 函数和 HAL 配置。
生成的代码可以用 \texttt{SimBackend} 直接运行和测试。

\section{技术创新点}

\subsection{声明式安全}

传统 PLC 编程中,安全互锁逻辑散落在梯形图的各个 rung 中,
审计时需要人工追踪所有相关的触点和线圈。
RustPLC 将安全约束提升为一等公民:

\begin{Verbatim}[frame=single,fontsize=\small]
safety: stamp_head.extended conflicts_with conveyor_motor.on
\end{Verbatim}

一行声明,编译器自动证明。审计者只需检查约束是否完整,不需要理解实现细节。

\subsection{物理感知编译}

传统编译器不理解物理世界。RustPLC 的编译器知道:

\begin{itemize}
  \item 电磁阀有响应时间(\texttt{response\_time: 15ms})
  \item 气缸有行程时间(\texttt{stroke\_time: 250ms})
  \item 信号沿因果链传播,每一跳都有延迟
\end{itemize}

这些物理参数直接参与时序验证和代码生成,不是注释或文档,而是编译器的输入。

\subsection{三模式统一}

同一份 DSL 源码、同一份生成代码,通过 \texttt{HalBackend} trait 的不同实现,
可以运行在三种完全不同的环境中:

\begin{enumerate}
  \item 纯内存仿真(CI/CD 流水线)
  \item 虚拟机集群(含电路和力学仿真的数字孪生)
  \item 真实硬件(FPGA + 继电器 + 传感器)
\end{enumerate}

开发者在模式 A 中快速迭代,在模式 B 中验证物理行为,在模式 C 中部署到产线。
整个过程不需要修改任何控制逻辑代码。

\section{未来工作}

\subsection{Phase 3 完成:Modbus 后端}

实现 \texttt{ModbusBackend},使用 \texttt{tokio-modbus} 0.17 库,
支持 Modbus RTU(串口)和 Modbus TCP(以太网)两种传输方式。
这是连接虚拟工厂和实体硬件的关键桥梁。

\subsection{Phase 4:虚拟 I/O 从站}

实现 Modbus slave daemon,集成 Gnucap 电路仿真和 jtufem-rs 力学仿真。
目标是在 196-ESXI-ARM 虚拟机集群上运行完整的数字孪生。

\subsection{Phase 5:FPGA 硬件}

在 iCESugar-pro 上实现 Verilog Modbus RTU slave 状态机,
通过 PMOD GPIO 驱动真实的继电器和传感器。

\subsection{长期愿景}

\begin{itemize}
  \item \textbf{模拟量 I/O}:支持 4--20mA 电流环和 0--10V 电压信号,
        实现 PID 控制回路。
  \item \textbf{多控制器协同}:多个 VM 运行不同的 PLC 程序,
        通过 Modbus 网络协同工作。
  \item \textbf{图形化 DSL 编辑器}:基于 Web 的可视化编辑器,
        拖拽设备、连线、配置约束,自动生成 \texttt{.plc} 文件。
  \item \textbf{OPC UA 集成}:支持 OPC UA 协议,
        与主流 SCADA/HMI 系统对接。
\end{itemize}

\section{成本与开源性}

RustPLC 的全部技术栈 100\% 开源(MIT/Apache 协议)。
硬件部分(模式 C)的总成本约 \$120:

\begin{table}[htbp]
\centering
\begin{tabular}{llr}
\toprule
\textbf{硬件} & \textbf{用途} & \textbf{价格} \\
\midrule
iCESugar-pro v1.3 & FPGA 开发板 & \textasciitilde\$50 \\
PMOD 继电器模块 & 数字输出 & \textasciitilde\$15 \\
PMOD 按钮/传感器 & 数字输入 & \textasciitilde\$15 \\
USB-UART 适配器 & Modbus RTU 通信 & \textasciitilde\$10 \\
杜邦线/面包板 & 接线 & \textasciitilde\$10 \\
\midrule
\textbf{合计} & & \textbf{\textasciitilde\$120} \\
\bottomrule
\end{tabular}
\caption{模式 C 硬件成本}
\end{table}

模式 A 和模式 B 不需要任何硬件投入。一台普通笔记本电脑即可运行完整的
编译验证流程和虚拟工厂仿真。

\vspace{2em}

\begin{center}
\rule{0.5\textwidth}{0.4pt}

\vspace{1em}

\textit{声明物理事实与安全意图,让编译器证明它是安全的;\\
然后生成确定性执行内核,驱动虚拟工厂或真实硬件。}

\vspace{1em}

\textbf{RustPLC} — 用 Rust 写的,所以它不会 panic。\\
好吧,至少不会在生产线上。
\end{center}

\end{document}
